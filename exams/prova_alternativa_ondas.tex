% Style.
\documentclass[letterpaper,portuguese,12pt,pdftex]{exam}

\usepackage{setspace}
\usepackage{lineno}
\usepackage[left=2.5cm,top=3cm,right=2.5cm]{geometry}

% Portuguese.
\usepackage[brazil]{babel}
\usepackage[T1]{fontenc}
\usepackage[utf8x]{inputenc}
\usepackage{textcomp}

% Font.
\usepackage{lmodern}

% Figures.
\usepackage{epsf,epsfig}

% Bibtex and extras.
\usepackage{natbib}
\usepackage{url}
\usepackage[bookmarks=false,colorlinks=true,urlcolor={green},linkcolor={green},pdfstartview={XYZ null null 1.22}]{hyperref}

% Math.
\usepackage{amssymb}
\usepackage{amsmath}
\usepackage{mathtools}
\usepackage{cancel}
\everymath{\displaystyle}

% Exam.
\addpoints
\printanswers
% \noprintanswers
\usepackage{color}
\definecolor{SolutionColor}{rgb}{0.8,0.9,1}
\shadedsolutions
\renewcommand{\solutiontitle}{\noindent\textbf{Solução:}\par\noindent}
\pagestyle{headandfoot}
\footer{}{Página \thepage\ de \numpages}{}
\boxedpoints
\pointsinrightmargin
\pointpoints{ponto}{pontos}
\hqword{Questão}
\hpword{Pontos}
\hsword{Nota}
% \qformat{\textbf{Question\thequestion}\quad(\thepoints)\hfill}

% User commands.
\newcommand{\pd}[2]{\dfrac{\partial #1}{\partial #2}}

% PDF metadata.
\pdfinfo{% hyperref overrides this
  /Title    (Prova 03 -- Ondas e Marés)
  /Author   (Filipe Fernandes)
  /Creator  (Filipe Fernandes)
  /Producer (Filipe Fernandes)
  /Subject  (prova)
  /Keywords (prova, oceanografia)
}

% Front page.
\title{Prova Alternativa -- Ondas e Marés}
\author{Prof. Filipe Fernandes}
\date{17-Dez-2012}

\begin{document}
\maketitle
\doublespacing

\vspace{1cm}
\hbox to \textwidth{Nome e número de matrícula:\enspace\hrulefill}
\vspace{1cm}

\begin{minipage}{.8\textwidth}
Esse exame incluí \numquestions\ questões. O número total de pontos é \numpoints.
\end{minipage}

\begin{questions}

\question[2]
  Defina e escreva a formulação matemática para {\bf número de onda} $K$,
  {\bf frequência} $\omega$ e {\bf velocidade de fase} $C$.  Essas fórmulas
  serão uma função de {\bf comprimento de onda} $\lambda$ e/ou {\bf período da
  onda} $T$.

  \begin{solution}
    \begin{align*}
      \omega &= \dfrac{2\pi}{T} \\
      K &= \dfrac{2\pi}{\lambda} \\
      C &= \dfrac{\lambda}{T} = \dfrac{\omega}{K}
    \end{align*}
  \end{solution}

\question
Sobre ondas longas (água rasa) e curtas (água profunda) responda.  Partindo da
{\bf relação de dispersão}
\[
  \omega^2 = gK\tanh(KH)
\]

\begin{parts}
  \part[2]
  Faça as aproximações para águas rasa (ondas longas) e águas profundas (ondas
  curtas).

\begin{solution}
\raggedright
      Água rasa (ondas longas) $KH << 1 \therefore \tanh(KH) \sim KH$\\
      \[
        \omega^2 = gK^2H
      \]

      Água profundas (ondas curtas) $KH >> 1 \therefore \tanh(KH) \sim 1$\\
      \[
        \omega^2 = gK
      \]
  \end{solution}

  \part[2]
  Qual a velocidade de uma onda de 20 m de comprimento que se propaga em
  águas profundas?

  \begin{solution}
    \[
      C = \sqrt{\dfrac{g\lambda}{2\pi}} = \sqrt{\dfrac{9.8 \times 20}{2\pi}} = 5.58 \text{ m s}^{-1}
    \]
  \end{solution}

\end{parts}

\question
 \begin{parts}
  \part[1]
  Dois observadores em bordas perpendiculares de uma piscina observam ondas que
  se propagam de forma retilínea e oblíqua às bordas da piscina.  Um observador
  mede que as cristas das ondas se propaga a 3 m s$^{-1}$, enquanto o outro
  mede 4 m s$^{-1}$.  Qual é a magnitude da velocidade de fase?  Essas são
  ondas longas ou curtas?

  \begin{solution}
  $C = (4^2 + 3^2)^{1/2} = 5$ m/s\\
  Não temos informação sobre $\lambda$ nem a profundidade da piscina para
  responder se essas são ondas longas ou curtas.
  \end{solution}

  \part[1]
  Se 30 cristas de onda passam, sucessivamente, em um ponto fixo durante um
  intervalo de tempo de 2 minutos e 8 segundos, qual é a frequência das ondas.

  \begin{solution}
    $\omega = \dfrac{30}{(2 \times 60)+8} = 0.234$ s$^{-1}$
  \end{solution}
\end{parts}

\question [3]
  Partindo da seguinte equação do modelo de águas rasas (sem rotação, nem
  limites com a costa).

  $$\dfrac{\partial^2 \eta}{\partial x^2} - \dfrac{1}{gH}\dfrac{\partial^2 \eta}
  {\partial t^2}$$

  Mostre que essa equação é satisfeita para ondas longas com a forma geral da
  superfície livre:

  $$\eta = \eta_o \cos(kx - \omega t).$$

  \begin{solution}
    \begin{align*}
      \dfrac{\partial^2}{\partial x^2}\left[ \eta_o\cos(kx-\omega t) \right] &=  \dfrac{1}{gH}\dfrac{\partial^2}{\partial t^2}\left[ \eta_o\cos(kx-\omega t) \right]\\
      -k\dfrac{\partial}{\partial x}\left[ \eta_o\sin(kx-\omega t) \right] &=  -\dfrac{\omega}{gH}\dfrac{\partial}{\partial t}\left[ \eta_o\sin(kx-\omega t) \right]\\
      -k^2\eta_o\cos(kx-\omega t) &= -\dfrac{\omega^2}{gH} \eta_o\cos(kx-\omega t)\\
      \dfrac{\omega^2}{gH} = k^2 &\therefore \omega^2 = gHk^2
    \end{align*}
  \end{solution}

\question
O sistema de equações abaixo representa o modelo de águas
rasas (equações \ref{eq:x-dir}, \ref{eq:y-dir} e \ref{eq:continuity}).

\begin{align}
  \pd{u}{t} + u\pd{u}{x} + v\pd{u}{y} &= +fv -g\pd{\eta}{x} \label{eq:x-dir} \\
  \pd{v}{t} + u\pd{v}{x} + v\pd{v}{y}&= -fu -g\pd{\eta}{y} \label{eq:y-dir} \\
  \pd{\eta}{t} + \pd{}{x}(Hu) + \pd{}{y}(Hv) &= 0 \label{eq:continuity}
\end{align}

 \begin{parts}
  \part[3]
  Em aula assumimos que o número de Rossby ($R_o$) e muito menor que 1.
  \[ R_o = \frac{U}{fL} << 1 \]

  Sabendo que $R_o$ é a escala da {\bf velocidade} sobre a escala {\bf horizontal
  de comprimento} divididos pelo {\bf parâmetro de Coriolis} quais expressões das
  equações \ref{eq:x-dir} e \ref{eq:y-dir} podemos eliminar?  O sistema de
  equações fica completamente linear?

  \begin{solution}
  \( R_o = \frac{U}{fL} << 1 \) significa que podemos cortar todos os termos
  advectivos.  Pois a escala da velocidade $U (u,v,uu,uv,vv)$ sobre a escala de
  tamanho $L (x,y)$ é muito menor que a frequência de Coriolis $f$
  \( \frac{U}{L} << f\).

  \begin{align*}
    \pd{u}{t} + \cancel{u\pd{u}{x}} + \cancel{v\pd{u}{y}} &= +fv -g\pd{\eta}{x} \\
    \pd{v}{t} + \cancel{u\pd{v}{x}} + \cancel{v\pd{v}{y}} &= -fu -g\pd{\eta}{y}
  \end{align*}
  \end{solution}

  \part[3]
  Assumindo $R_o << 1$ nos permitiu eliminar os termos não lineares das equações
  do movimento nas direções $x$ e $y$ (\ref{eq:x-dir} e \ref{eq:y-dir}).  Porém,
  a equação da continuidade (\ref{eq:continuity}) ainda é não-linear.

  Identifique o termo não linear e explique como aproximação abaixo nos
  permite linearizar o problema.

  \begin{align*}
    \eta &<< H\\
    b &<< h
  \end{align*}

  \begin{solution}
   A aproximação de ondas de pequena amplitude ($\eta << H$) diz que a elevação
   do nível do mar é muito menor que a coluna d'água.  Isso nos permite dizer
   que $H$ não varia no tempo com $\eta$.

   A aproximação feita com a batimetria ($b<<h$) nos diz que a variação
   batimétrica é muito menor que á média da coluna d'água.  Isso nos permite
   ignorar as variações da topografia de fundo, logo $h \sim H$.  Agora podemos
   simplificar a equação da continuidade para a forma linear assumindo um $h$
   constante:
   \[ \pd{\eta}{t} = -h\left( \pd{u}{x} + \pd{v}{y}\right)\]
  \end{solution}

  \part[2]
  Nós também assumimos o número de Rossby temporal ($R_{oT}$) como
  sendo aproximadamente 1:
  \[ R_{oT} = \frac{1}{fT} \sim 1 \]

  O que essa aproximação significa?  Como era o Rossby temporal das ondas curtas
  e longas {\bf não} influenciadas pela rotação da Terra?

  \begin{solution}
   Quando o número de Rossby temporal é aproximadamente ``1'' a rotação da
   Terra se torna importante na escala em estudo pois a frequência do movimento
   $\frac{1}{T}$ é aproximadamente igual à frequência de Coriolis $f$.
   \[ \frac{1}{T} \sim f \]

   Nas ondas longas e curtas não influenciadas pela rotação da Terra nos temos
   que a frequência do movimento é muito maior que a frequência de Coriolis.
   \( \frac{1}{T} >> f \), ou \( R_{oT} >> 1 \)
  \end{solution}
 \end{parts}

\question
 Em aula nos vimos ondas 3 tipos de ondas influenciadas pela rotação da Terra:
 ondas de Poincaré, Kelvin e Rossby.  Todas derivadas do nosso
 modelo de águas rasas.  Porém, cada uma assume diferente premissas para
 simplificar o problema e resolver a onda em questão.

\begin{parts}
 \part[1]
 Como as ondas de Kelvin e Poincaré se diferenciam das ondas de Rossby?\\
 (Dica: $f = f_o + \beta y$)

 \begin{solution}
  Ondas de Poincaré e Kelvin são resolvidas no plano-$f$ ($f_o$) ou seja, o
  efeito da rotação é simplificado para primeira ordem como sendo uma expansão
  de Taylor ao redor da latitude.
  \[f \sim f_o = 2\Omega\sin\phi_o \]
  Isso ignora os efeitos da esfericidade da Terra (representada pelo
  plano-$\beta$), essencial para a existências das ondas de Rossby.
 \end{solution}

  \part[1]
  A onda de Kelvin é um caso especial de onda de Poincaré que viaja sempre com
  um contorno do seu lado esquerdo (direito) no hemisfério sul (norte).  Dito
  isso, simplifique as equações do nosso modelo de águas rasas (\ref{eq:x-dir},
  \ref{eq:y-dir} e \ref{eq:continuity}) para uma ondas de Kelvin.

\vspace{2cm}
\begin{solution}
  \begin{align*}
  \cancelto{0}{\pd{u}{t}} &= +fv -g\pd{\eta}{x}\\
  \pd{v}{t} &= \cancelto{0}{-fu} -g\pd{\eta}{y}\\
  \pd{\eta}{t} &= -h\left(\cancelto{0}{\pd{u}{x}} + \pd{v}{y}\right)
  \end{align*}
\end{solution}

  \part[1]
  A relação de dispersão das ondas de Poincaré é:
  \[
    \omega^2 = f_o^2 + ghk^2
  \]

  Discuta o extremo onde:

  \[
    \omega >> f_o
  \]

\begin{solution}
\raggedright
  $\omega >> f_o \rightarrow$ Temos de volta a relação de dispersão de ondas de
  água rasa (ondas longas).\\
\end{solution}
\end{parts}

\question
Ondas de vorticidade (ou ondas Rossby).

Para descrever as ondas de Rossby em aula nós utilizamos a forma completa de $f$.

\begin{align*}
  \pd{u}{t} &= (f_o + \beta y)v - g\pd{\eta}{x}\\
  \pd{v}{t} &= -(f_o + \beta y)u - g\pd{\eta}{y}\\
  \pd{\eta}{t} &= -h\left(\pd{u}{x} + \pd{v}{y}\right)
\end{align*}

\begin{parts}
\part[2]
Explique o que a aproximação do plano $\beta$ representa.

\begin{solution}
 O plano $\beta$ é a aproximação para representar a esfericidade da Terra em
 coordenadas cartesianas.
 \[ \beta = \frac{2\Omega\cos(\phi_o)}{R}\]
\end{solution}

\part[1]
A relação de dispersão de ondas de Rossby é dada por:
\[\omega = \frac{-\beta k}{k^2 + l^2 + \frac{f_o^2}{gh}}\]

Prove que essas ondas se propagam energia apenas para Oeste.

\begin{solution}
 \[
    C \equiv \frac{\omega}{k} = \frac{-\beta}{k^2 + l^2 + \frac{f_o^2}{gh}}
 \]
 $C$ é sempre negativo, ou seja se propaga na direção $-x$, Oeste no nosso
 sistema de coordenadas.
\end{solution}

\end{parts}

\end{questions}
\end{document}
