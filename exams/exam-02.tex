% Style.
\documentclass[letterpaper,portuguese,12pt,pdftex]{exam}

\usepackage{setspace}
\usepackage{lineno}
\usepackage[left=2.5cm,top=3cm,right=2.5cm]{geometry}

% Portuguese.
\usepackage[brazil]{babel}
\usepackage[T1]{fontenc}
\usepackage[utf8x]{inputenc}
\usepackage{textcomp}

% Font.
\usepackage{lmodern}

% Figures.
\usepackage{epsf,epsfig}

% Bibtex and extras.
\usepackage{natbib}
\usepackage{url}
\usepackage[bookmarks=false,colorlinks=true,urlcolor={green},linkcolor={green},pdfstartview={XYZ null null 1.22}]{hyperref}

% Math.
\usepackage{amssymb}
\usepackage{amsmath}
\usepackage{mathtools}
\usepackage{cancel}
\everymath{\displaystyle}

% Exam.
\addpoints
\printanswers % \noprintanswers
\usepackage{color}
\definecolor{SolutionColor}{rgb}{0.8,0.9,1}
\shadedsolutions
\renewcommand{\solutiontitle}{\noindent\textbf{Solução:}\par\noindent}
\pagestyle{headandfoot}
\footer{}{Página \thepage\ de \numpages}{}
\boxedpoints
\pointsinrightmargin
\pointpoints{ponto}{pontos}
\hqword{Questão}
\hpword{Pontos}
\hsword{Nota}
% \qformat{\textbf{Question\thequestion}\quad(\thepoints)\hfill}

% User commands.
\newcommand{\pd}[2]{\dfrac{\partial #1}{\partial #2}}

% PDF metadata.
\pdfinfo{% hyperref overrides this
  /Title    (Prova 02 -- Ondas e Marés)
  /Author   (Filipe Fernandes)
  /Creator  (Filipe Fernandes)
  /Producer (Filipe Fernandes)
  /Subject  (prova)
  /Keywords (prova, oceanografia)
}

% Front page.
\title{Prova 02 -- Ondas e Marés}
\author{Prof. Filipe Fernandes}
\date{06-Dez-2013}

\begin{document}
\maketitle
\doublespacing

\vspace{2cm}
\hbox to \textwidth{Nome e número de matrícula:\enspace\hrulefill}
\vspace{2cm}

\begin{minipage}{.8\textwidth}
Esse exame incluí \numquestions\ questões. O número total de pontos é \numpoints.
\vspace{1cm}

Essa prova segue o Acordo Ortográfico da Língua Portuguesa de 1990 (em vigor no
início de 2009).  Por isso erros de ortografia e gramática serão descontados da
sua nota final.

A prova deve ser feita individualmente e sem consulta.  O aluno deverá usar
CANETA (preta ou azul) para responder as questões – qualquer questão respondida
à lápis não será considerada na hora da correção.  Coloque seu nome em TODAS as
folhas e numere as mesma colocando o número total de folhas (Ex.: 1/4, 2/4 e
etc).

\vspace{1cm}

Leia atentamente todas as questões: a interpretação faz parte da prova e dúvidas
serão esclarecidas apenas após o término da mesma.

\end{minipage}

\clearpage
% 15
\begin{questions}
\question
Sobre ondas internas e externas.

\begin{parts}
\part[1]
Ondas externas são oscilações entre a interface ar-mar, já as ondas internas são
oscilações na interface entre densidades.  Sabendo que, quanto maior a diferença
de densidade mais energia será necessária para mudar o estado do sistema, onde
as oscilações são mais facilmente estabelecidas? Em uma interface interna ou na
superfície do mar? Justifique a sua resposta.

\begin{solution}
As oscilações são mais facilmente estabelecidas em uma interface interna do
que na superfície do mar.  A diferença de densidade entre duas camadas de água é
menor do que entre água e ar.  Densidade do ar = 1,2 kg m$^{-3}$ e densidade da
água do mar = 1025 kg m$^{-3}$.  Consequentemente menos energia é requerida para
gerar ondas internas do que ondas de superfície de similar amplitude.
\end{solution}


\part[1]
Sabemos que processos de mistura vertical são cruciais para a vida marinha,
tornando nutrientes mais acessíveis a biota.  Dito isso, quais ondas são mais importantes no contexto de processos de mistura vertical: ondas de superfície ou
ondas internas? Justifique a sua resposta.

\begin{solution}
Ondas internas são mais importantes no contexto de processos de mistura
vertical, especialmente quando quebram e levam nutrientes da região abaixo da
picnoclina para a base da camada de mistura.
\end{solution}
\end{parts}

\question
A relação de dispersão de ondas de gravidade longas (águas rasas) é dada por:

\[
  \omega = k\sqrt{gh},
\]

Sabendo que a velocidade de fase dessas ondas é $C=\sqrt{gh}$ podemos
re-escrevê-la na forma:

\[
  \omega = kC,
\]


\begin{parts}
\part[1]
Explique como o lado direito da equação se relaciona com o lado esquerdo, ou
seja, como frequência angular de relaciona com número de onda. (A essência de
qualquer relação de dispersão!).  Faça a mesma consideração em termos de período
$T$ e comprimento de ondas $L$.

\begin{solution}
\[
  \dfrac{\omega}{k} = \sqrt{gh},
\]
Quanto maior a frequência (menor o período) mais rápidas serão as OL.  Já
quanto menor o número de onda (maior o comprimento), mais lentas serão as OL.
\end{solution}


\part[1]
Nas ondas longas acima nós assumimos que o número de Rossby temporal ($R_{oT}$)
como sendo aproximadamente muito maior que 1:
\[
  R_{oT} = \frac{1}{fT} >> 1
\]

Rearranjando essa equação e substituindo o período $T$ pela frequência $\omega$
(lembrando que $T = \dfrac{2\pi}{\omega}$) temos:

\[
  \dfrac{\omega}{2\pi} >> f, \text{ ou } \omega >> f
\]

Nossa velha conhecida, dizendo que as ondas de gravidade têm frequência angular
(período) muito maior (muito menor) que a frequência (período) inercial.


Agora imaginem uma onda muito longa com frequência muito baixa (Ex.: as ondas de
Rossby), como seria a relação entre $\omega$ e $f$ para elas?  Podemos descartar
$f$ como fizemos nas Ondas de gravidade de superfície?

\begin{solution}
Agora as ondas possuem frequência angular (período) muito menor (maior) que a
frequência (período) inercial sendo:
  $\omega << f$
\end{solution}

\end{parts}


\question
Em aula nos vimos ondas 3 tipos de ondas influenciadas pela rotação da Terra:
ondas de Poincaré, Kelvin e Rossby.  Todas derivadas do nosso modelo de águas
rasas.  Porém, cada uma assume diferente premissas para simplificar o problema e
resolver a onda em questão.

\begin{parts}
\part[1]
Como as ondas de Kelvin e Poincaré se diferenciam das ondas de Rossby?\\
(Dica: lembre-se que $f = f_o + \beta y$)

\begin{solution}
Ondas de Poincaré e Kelvin são resolvidas no plano-$f$ ($f \equiv f_o$) ou seja, o
efeito da rotação é simplificado para primeira ordem como sendo uma expansão
de Taylor ao redor da latitude.
\[f \sim f_o = 2\Omega\sin\phi_o \]
Isso ignora os efeitos da esfericidade da Terra (representada pelo
plano-$\beta$), essencial para a existências das ondas de Rossby.
\end{solution}

\part[1]
A relação de dispersão das ondas de Poincaré é,
\[
  \omega = f_o + k\sqrt{gh}
\]

e de ondas de Kelvin é,
\[
  \omega = -l\sqrt{gh}
\]

Discuta a diferença e semelhança entre essas duas em usando a equação e explique
a afirmação:

``A onda de Kelvin pode ser considerada um caso especial da onda de Poincaré que,
viaja sempre com um contorno do seu lado esquerdo (direito) no hemisfério sul
(norte).''
(Dica: $k$ é o número de onda em $x$, $l$ é o número de onda em $y$.)

\begin{solution}
  A onda de Kelvin é uma onda ``sem a influência da rotação da Terra'' em
  ``$y$'', mas com a influência da Rotação da Terra em seu balanço Geostrófico
  em ``$x$''.  Já a Poincaré pode ser vista como uma Onda de Kelvin ``Sem essa
  restrição assimétrica na horizontal'', sendo isotrópica e com a influência da
  rotação em ambos ``$x$'' e ``$y$'' igualmente.
\end{solution}

\part[3]
Usando a relação de dispersão de ondas de Poincaré discuta os extremos onde:

\begin{align*}
  \omega &>> f_o \text{ e,}\\
  \omega &\sim f_o
\end{align*}

\begin{solution}
$\omega >> f_o.$
Temos de volta a relação de dispersão de ondas de água rasa.
$\frac{\omega}{k} = \sqrt{gh}$

$\omega \sim f_o \therefore 0 = ghk^2.$ Como sabemos que $g$ e $h$ não são zero,
temos $k=0$, ou seja, não há ondas nesse limite.
\end{solution}

\part[2]
A onda de Kelvin, como já vimos acima, precisa de um ``contorno'' para se
propagar.  Em geral, essa ondas são a manifestação das marés na plataforma
costeira.  Porém, temos um caso especial de ondas de Kelvin chamada de ondas
de Kelvin Equatorial.  Essas ondas são responsáveis pelo fenômeno conhecido
como {\it El-Niño}.

Explique como esse tipo de onda pode existir.  Use
$f = f_o + \beta y$ em sua explicação.\\
(Dica: Qual é o valor de $f_o$ no equador?)

\begin{solution}
Como $f = f_o + \beta y$, apesar de $f_o = 0$, $f$ não é zero, pois ainda tem
o efeito $\beta$.  Já o ``contorno'' necessário para sua existência é a forte
variação de $f$ ao redor do Equador, pois sai de próximo a zero para um valor
significativo ($\pm 5^{\circ}$ graus de latitude).
\end{solution}

\end{parts}

\question
Ondas de vorticidade (ou ondas Rossby) tem a relação de dispersão definida por:

\[
  \omega = \dfrac{-\beta k}{k^2 + l^2 + \dfrac{1}{R^2}}
\]

Sabendo que o comprimento de onda $L$ é dado por $\mathbf{k} = \dfrac{2\pi}{L}$,
e $\mathbf{k} = \sqrt{k^2 + l^2}$. (Lembre-se que $\mathbf{k}$ é o vetor número.)

\begin{parts}

\part[1]
Faça as considerações para Ondas de Rossby curtas ($L << R$, comprimento de
onda menor que raio de deformação) e Ondas de Rossby longas ($L >> R$,
comprimento de onda maior que o raio de deformação). (Lembre-se que $R$ é o
raio de deformação de Rossby e é definido por $R = \dfrac{\sqrt{gh}}{f_o}$).

Dica: com algumas manipulações algébricas podemos re-arranjar a relação de
dispersão para ficar na forma de,

  \[
    \omega = -\beta k \left( \dfrac{L^2 R^2}{R^2 + L^2} \right)
  \]


  \begin{solution}
  Onda curta (quando $L << R$),
  \begin{align*}
    \omega &= -\beta k \left( \dfrac{L^2 R^2}{R^2 + \cancelto{<< R^2}{L^2}} \right) \\
    \omega &= -\beta k \left( \dfrac{L^2 \cancel{R^2}}{\cancel{R^2}} \right) \\
    \omega &\approx -\beta k L^2
  \end{align*}

  Onda longa (quando $L >> R$),
  \begin{align*}
    \omega &= -\beta k \left( \dfrac{L^2 R^2}{\cancelto{<< L^2}{R^2} + L^2} \right) \\
    \omega &= -\beta k \left( \dfrac{\cancel{L^2} R^2}{\cancel{L^2}} \right) \\
    \omega &\approx -\beta k R^2
  \end{align*}
\end{solution}

\part[1]
A relação de dispersão de Ondas de Rossby Longas do item acima relaciona
$\omega$, $k$, $\beta$ e $R^2$, enquanto a relação de dispersão das Ondas de
Rossby Curtas relaciona $\omega$, $k$ e $L$.

Assumindo que as Ondas de Rossby tem a dimensão em $x$ muito maior que a
dimensão em $y$ (o que significa $l\equiv0$), como podemos simplificar ainda
mais a relação de dispersão de Ondas de Rossby Curtas?

\begin{solution}
  \begin{align*}
    \omega &\approx -\beta k L^2 \\
    \omega &\approx -\beta \cancel{k} \dfrac{2\pi}{k\cancel{^2} + \cancelto{0}{l^2}} \\
    \omega &\approx -\dfrac{\beta}{k} \\
  \end{align*}
\end{solution}


\part[1]
Usando a forma simplificada da relação de dispersão para Ondas Curtas encontrada
acima calcule a velocidade de fase $\left( C = \dfrac{\omega}{k} \right)$ e a
velocidade de grupo $\left( Cg = \dfrac{\partial\omega}{\partial k} \right)$.

\begin{solution}
  A velocidade de fase da ORC:
  \begin{align*}
    \omega &\approx -\dfrac{\beta}{k} \div k\\
    C = \dfrac{\omega}{k} &\approx -\dfrac{\beta}{k^2}
  \end{align*}

  A velocidade de grupo da ORC:
  \begin{align*}
    \dfrac{\partial \omega}{\partial k} &\approx \dfrac{\partial (-{\beta}k^{-1})}{\partial k} \\
    Cg = \dfrac{\partial \omega}{\partial k} &\approx -(-\beta){k^{-2}} \\
    Cg &\approx = \dfrac{\beta}{k^2}
  \end{align*}
\end{solution}

\part[1]
Faça o mesmo para as Ondas de Rossby Longas.

\begin{solution}
  A velocidade de fase da ORL:
  \begin{align*}
    \omega &\approx -\beta k R^2 \div k\\
    C = \dfrac{\omega}{k} &\approx -\beta R^2
  \end{align*}

  A velocidade de grupo ORL:
  \begin{align*}
    \dfrac{\partial \omega}{\partial k} &\approx \dfrac{\partial(-\beta k R^2)}{\partial k}\\
    Cg = \dfrac{\partial \omega}{\partial k} &\approx -\beta R^2\\
  \end{align*}
\end{solution}

\part
Extra (5 pontos):
Sabemos que as correntes oceânicas são intensificadas no contorno oeste e que as
Ondas de Rossby tem um papel fundamental nesse fenômeno.  Explique o papel das
Ondas de Rossby nesse efeito usando $C$ e $Cg$ calculados acima.  Porque o mesmo
não ocorre na atmosfera?

(Dica: O que acontece com a onda de Rossby ao encontrar uma barreira?)

\begin{solution}
As Ondas de Rossby Longas viajam com fase e grupo para Oeste ($Cg = -\beta R^2$),
mas ao encontrar se aproximarem do continente essa ondas Longas se tornam ondas
curtas (razão de aspecto em ``$x$'' é muito menor que em ``$y$'' próximo à
costa) e propagam fase para Oeste $\left( C = -\dfrac{\beta}{k^2} \right)$ e
grupo para Leste $\left( C = \dfrac{\beta}{k^2} \right)$.   As ondas curtas são
então dispersivas e se acumulam no contorno Oeste, logo, sua energia fica
acumulada nos contornos dos continentes.  Já na atmosfera não há contornos, logo
não á intensificação.
\end{solution}

\end{parts}

\end{questions}
\end{document}
