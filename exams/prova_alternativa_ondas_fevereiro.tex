% Style.
\documentclass[letterpaper,portuguese,12pt,pdftex]{exam}

\usepackage{setspace}
\usepackage{lineno}
\usepackage[left=2.5cm,top=3cm,right=2.5cm]{geometry}

% Portuguese.
\usepackage[brazil]{babel}
\usepackage[T1]{fontenc}
\usepackage[utf8x]{inputenc}
\usepackage{textcomp}

% Font.
\usepackage{lmodern}

% Figures.
\usepackage{epsf,epsfig}

% Bibtex and extras.
\usepackage{natbib}
\usepackage{url}
\usepackage[bookmarks=false,colorlinks=true,urlcolor={green},linkcolor={green},pdfstartview={XYZ null null 1.22}]{hyperref}

% Math.
\usepackage{amssymb}
\usepackage{amsmath}
\usepackage{mathtools}
\usepackage{cancel}
\everymath{\displaystyle}

% Exam.
\addpoints
% \printanswers
\usepackage{color}
\definecolor{SolutionColor}{rgb}{0.8,0.9,1}
\shadedsolutions
\renewcommand{\solutiontitle}{\noindent\textbf{Solução:}\par\noindent}
\pagestyle{headandfoot}
\footer{}{Página \thepage\ de \numpages}{}
\boxedpoints
\pointsinrightmargin
\pointpoints{ponto}{pontos}
\hqword{Questão}
\hpword{Pontos}
\hsword{Nota}
% \qformat{\textbf{Question\thequestion}\quad(\thepoints)\hfill}

% User commands.
\newcommand{\pd}[2]{\dfrac{\partial #1}{\partial #2}}

% PDF metadata.
\pdfinfo{% hyperref overrides this
  /Title    (Prova Alternativa -- Ondas e Marés)
  /Author   (Filipe Fernandes)
  /Creator  (Filipe Fernandes)
  /Producer (Filipe Fernandes)
  /Subject  (prova)
  /Keywords (prova, oceanografia)
}

% Front page.
\title{Prova Alternativa -- Ondas e Marés}
\author{Prof. Filipe Fernandes}
\date{23-Fev-2014}

\begin{document}
\maketitle
\doublespacing

\vspace{1cm}
\hbox to \textwidth{Nome e número de matrícula:\enspace\hrulefill}
\vspace{1cm}

\begin{minipage}{.8\textwidth}
Esse exame incluí \numquestions\ questões. O número total de pontos é \numpoints.
\end{minipage}

\begin{questions}

\question[8]
Partindo das {\bf três componentes da equação do movimento} nas direções
$x$, $y$ e $z$,

\begin{align*}
x: \pd{u}{t} &+ u\pd{u}{x} &+ v\pd{u}{y} &+ w\pd{u}{z} &= &+ fv &- \dfrac{1}{\rho}\pd{p}{x}         &+ \dfrac{1}{\rho}\left(\pd{\tau^{xx}}{x} + \pd{\tau^{xy}}{y} + \pd{\tau^{xz}}{z}\right) \\
y: \pd{v}{t} &+ u\pd{v}{x} &+ v\pd{v}{y} &+ w\pd{v}{z} &= &- fv &- \dfrac{1}{\rho}\pd{p}{y}         &+ \dfrac{1}{\rho}\left(\pd{\tau^{xy}}{x} + \pd{\tau^{yy}}{y} + \pd{\tau^{yz}}{z}\right) \\
z: \pd{w}{t} &+ u\pd{w}{x} &+ v\pd{w}{y} &+ w\pd{w}{z} &= &+  0 &- \dfrac{1}{\rho}\pd{p}{z} -\rho g &+ \dfrac{1}{\rho}\left(\pd{\tau^{xz}}{x} + \pd{\tau^{yz}}{y} + \pd{\tau^{zz}}{z}\right),
\end{align*}

responda:
\begin{itemize}
  \item Explique {\bf cada termo} das equações;
  \item O sistema se encontra fechado e com solução possível caso
        consideremos a densidade constante e o atrito desprezível?
  \item Comente sobre as aproximações usadas para chegar na equação de
        {\bf Laplace} para as pressões ($\nabla^2 \tilde{p}= 0$).
        {\bf Comente e explique} pelo menos 5.
\end{itemize}


\begin{solution}
  \begin{itemize}
    \item Não, mesmo assim teremos 3 equações e 4 incógnitas ($u$, $v$, $w$ e
    $p$), precisamos da equação da continuidade
    $\left( \pd{u}{x} + \pd{v}{y} + \pd{w}{z} = 0 \right)$ para fechar o sistema.
    \item ``locais, advectivos, Coriolis, FGP, FA e g''
    \item ``Apostila''
  \end{itemize}
\end{solution}

\question[4]
Partindo da {\bf relação de dispersão},
\[
  \omega^2 = gK\tanh(KH),
\]
faça as aproximações para {\bf águas rasa} (ondas longas) e {\bf águas profundas}
(ondas curtas). (Mostre o seu desenvolvimento!)

\begin{solution}
\raggedright
      Água rasa (ondas longas) $KH << 1 \therefore \tanh(KH) \sim KH$\\
      \[
        \omega^2 = gK^2H
      \]

      Água profundas (ondas curtas) $KH >> 1 \therefore \tanh(KH) \sim 1$\\
      \[
        \omega^2 = gK
      \]
  \end{solution}

\question
Você está viajando em um barco na direção paralela à propagação do que parece
ser ondas lineares no oceano.  Enquanto você observa as cristas passarem pelo
navio você nota 7 cristas (ou seja, 6 ondas) passarem num intervalo de 3
minutos.

Ao checar com a casa de máquinas você descobre que o navio está se movendo
numa velocidade de 5 nós (2.57 m s$^{-1}$), e a eco-sonda marca 10 m de profundidade
quase constante.

\begin{parts}
  \part[2\half]
  Com a informação acima calcule o período de onda relativo ao barco ($T'$).
  Escreva uma relação para o comprimento de onda relativo ($L'$). Calcule um
  valor para esse comprimento de onda ``percebido''.
  (Dica: $L'$ é o quão longe o navio viaja de uma crista a outra)

  \begin{solution}
    \begin{align*}
      T' &= \dfrac{3 \times 60\text{ s}}{6} \\
      T' &= 30 \text{ s} \\
      L' &= UT'  \\
      L' &= 2.57 \text{ m s}^{-1} \times 30 \text{ s} \\
      L' &= 77 \text{ m}
    \end{align*}
  \end{solution}

  \part[2\half]
  Crie uma expressão para o período real da onda (sobre um ponto de referência
  da Terra e não do barco).  Essa expressão deve ser em função de $T', L'$ e $L$
  (que ainda não sabemos).

  (Dica: Considere a fração de um comprimento de onda completo que se move
  durante o período percebido.)

  \begin{solution}

    \begin{center}
      \includegraphics[scale=0.8]{../exercicios/figures/boat_wave.png}
    \end{center}

    \begin{align*}
      \Delta T &= T_2 - T_1 \\
      \Delta X &= L - L' \\
      \dfrac{\Delta X}{\Delta T} &= \frac{L}{T} = \frac{L - L'}{T'} \\
      T &= T'\left(\frac{L}{L - L'}\right)
    \end{align*}
  \end{solution}

  \part[1\half]
  Usando a sua resposta acima, juntamente com a relação de dispersão, resolva
  o período de onda $T$.  Simplifique assumindo ondas  longas.  Quantas
  soluções válidas existem?

  (Dica: Considere a fração de um comprimento de onda completo que se move
  durante o período percebido.)

  \begin{solution}
    \begin{align*}
      \omega^2 &= k^2hg \\
      \left( \dfrac{2\pi}{T} \right)^2 &= \left( \dfrac{2\pi}{L} \right)^2 gh \\
      L &= T\sqrt{gh}
    \end{align*}

    Combinando com o resultado acima:

    \begin{align*}
      T &= T'\pm \frac{L'}{\sqrt{gh}} \\
      T &= 30\text{ s} \pm \frac{77\text{ m}}{\sqrt{9.8 \text{ m s}^{-2} 10\text{ m}}} \\
      T &= 30 \text{ s} \pm 7.77\text{ s} \\
      T &= 37.8 \text{ s ou } 22.2\text{ s}
    \end{align*}

    Com as informações fornecidas não há como inferir a direção de propagação
    dessas ondas.
  \end{solution}


  \part[1\half]
  Ache o comprimento de ondas associada a cada $T$.

  \begin{solution}
    \begin{align*}
      L &= T\sqrt{gh} \\
      L_1 &= 37.8\text{ s } \times 9.9\text{ m s}^{-1} = 374\text{ m} \\
      L_2 &= 22.2\text{ s } \times 9.9\text{ m s}^{-1} = 220\text{ m}
    \end{align*}
  \end{solution}

  \part[2]
  Avalia a sua solução sobre a consideração de ondas longas que fizemos.  A
  solução é válida?

  \begin{solution}
    Ambas são validas $\dfrac{h}{L} < 0.05$.
    \begin{align*}
      \frac{h}{L_1} = 0.027 \\
      \frac{h}{L_2} = 0.045
    \end{align*}
  \end{solution}

  \end{parts}

\question[2]
Sabemos que processos de {\bf mistura vertical} são cruciais para a vida
marinha, tornando nutrientes mais acessíveis a biota.  Discorra sobre o papel
das ondas internas na mistura vertical.  (Dica: Lembre-se que o oceano pode ser
divido em 3 extratos: camada de mistura, picnoclina e camada profunda.  Nesse
contexto onde estão plankton, as ondas internas e os nutrientes?)

\begin{solution}
  As ondas internas estão geralmente em zonas de alto gradiente de densidade
  (interface interna) como a picnoclina.  Quando há ondas internas nessa região,
  seu movimento vertical pode disponibilizar nutrientes da camada profunda logo
  abaixo para a camada de mistura acima.  Isso ocorre principalmente quando há
  quebra da onda interna, promovendo uma mistura entres essas interfaces.
  Disponibilizando assim nutrientes que antes estavam inacessíveis ao plankton da camada de mistura.
\end{solution}

\question
 Em aula nos vimos ondas 3 tipos de ondas influenciadas pela rotação da Terra:
 ondas de Poincaré, Kelvin e Rossby.  Todas derivadas do nosso
 modelo de águas rasas.  Porém, cada uma assume diferente premissas para
 simplificar o problema e resolver a onda em questão.

\begin{parts}
 \part[1]
 Como as ondas de Kelvin e Poincaré se diferenciam das ondas de Rossby?

 \begin{solution}
  Ondas de Poincaré e Kelvin são resolvidas no plano-$f$ ($f_o$) ou seja, o
  efeito da rotação é simplificado para primeira ordem como sendo uma expansão
  de Taylor ao redor da latitude.
  \[f \sim f_o = 2\Omega\sin\phi_o \]
  Isso ignora os efeitos da esfericidade da Terra (representada pelo
  plano-$\beta$), essencial para a existências das ondas de Rossby.
 \end{solution}

  \part[1]
  A onda de Kelvin é um caso especial de onda de Poincaré que viaja sempre com
  um contorno do seu lado esquerdo (direito) no hemisfério sul (norte).  Dito
  isso, explique o  que deve ser feito para aproximarmos as equações de um
  modelo de águas rasas para uma ondas de Kelvin.

\begin{solution}
  \begin{align*}
    \cancelto{0}{\pd{u}{t}} &= +fv -g\pd{\eta}{x}\\
    \pd{v}{t} &= \cancelto{0}{-fu} -g\pd{\eta}{y}\\
    \pd{\eta}{t} &= -h\left(\cancelto{0}{\pd{u}{x}} + \pd{v}{y}\right)
  \end{align*}
\end{solution}

  \part[1]
  A relação de dispersão das ondas de Poincaré é:
  \[
    \omega^2 = f_o^2 + ghk^2
  \]

  Discuta o extremo onde:

  \[
    \omega >> f_o
  \]

\begin{solution}
\raggedright
  $\omega >> f_o \rightarrow$ Temos de volta a relação de dispersão de ondas de
  água rasa (ondas longas).\\
\end{solution}
\end{parts}

\question[2]
Explique as afirmativas:
\begin{itemize}
  \item[a)] A natureza da ondas de Kelvin, de ser uma onda em uma direção e um
        balanço geostrófico em outra, resulta nos pontos anfidrômicos de maré
        nos oceanos.
  \item[b)] A teoria da maré de equilíbrio não é suficiente para explicar a
            manifestação da maré na plataforma costeira.
\end{itemize}

\question[3]
Você foi contrato pela empresa ACME\circledR{} para executar uma análise de
marés na costa da praia de {\it Lost}.  Sabendo que o número de forma da maré é uma
razão entre as principais constituintes {\bf diurnas} sobre a
{\bf semi-diurnas}. Calcule o {\bf Número de Forma} (F) para essa praia dados as
amplitudes abaixo (em metros) e comente sobre a predominância diurna ou
semi-diurna no local.  Discorra sobre como isso pode lhe ajudar a caracterizar
a dinâmica do local.

\begin{center}
\begin{tabular}{ll}
Constituinte & Amplitude (m)\\
\hline
K1 & 0,7 \\
O1 & 0,2 \\
M2 & 2,4 \\
S2 & 1,2 \\
\hline
\end{tabular}
\end{center}


\begin{solution}
  \begin{align*}
    \text{F} &= \dfrac{\text{K1 + O1}}{\text{M2 + S2}} \\
    \quad \\
    \text{F} &= \dfrac{0,7 + 0,2}{2,4 + 1,2} = 0.25
  \end{align*}

  Maré predominantemente semi-diurna.
\end{solution}

\question[3]
Você foi contrato novamente pela empresa ACME\circledR{} para avaliar o
potencial energético da praia acima.  Eles querem saber se, o potencial
energético da maré é mais eficiente que o potencial das ondas de gravidade.

\begin{itemize}
  \item[a)] Quais são as variáveis devemos medir para responder essa pergunta?
  \item[b)] Qual seria o maior potencial energético?  Ondas ou marés? Explique.
\end{itemize}

(Dica: A equação abaixo mostra como calcular a densidade de energia [J m$^{-2}$]
das ondas de gravidade externas.)

\[
  E = \dfrac{\rho g}{8} H^2, \text{ onde } \dfrac{\rho g}{8} \approx 1,25 \text{ kg m}^{-2}\text{s}^{-2}
\]

\begin{solution}
  Devemos medir a altura de ondas por longos períodos (que compreendam toda a
  variabilidade local das mesmas) para estimar com alguma segurança.  A
  propriedade estatística $H_{1/3}$ ajuda a reduzir o número de observações e
  ainda assim adquirir uma estimativa robusta.

  Para sabermos se as ondas de gravidade geradas pelo vento serão ou não mais
  eficientes que a maré, podemos utilizar uma estimativa baseado apenas na
  constituinte mais significativa (M2).

  $E = 1,25 \times 2,4^2 = 7.2$ J m$^{-2}$, duas vezes por dia seria
  14.4 J m$^{-2}$ por dia.

  Mesmo uma onda pequena (30 cm), com um período típico de 10 s forneceria mais
  energia que a maré acima.

  $E = 1,25 \times 0.3^2 = 0.112$ J m$^{-2}$, 4320 vezes por dia
  ($24 \times60 \times 30 / 20$) seria 486 J m$^{-2}$ por dia.
\end{solution}

\end{questions}
\end{document}
