% Style.
\documentclass[letterpaper,portuguese,12pt,pdftex]{exam}

\usepackage{setspace}
\usepackage{lineno}
\usepackage[left=2.5cm,top=3cm,right=2.5cm]{geometry}

% Portuguese.
\usepackage[brazil]{babel}
\usepackage[T1]{fontenc}
\usepackage[utf8x]{inputenc}
\usepackage{textcomp}

% Font.
\usepackage{lmodern}

% Figures.
\usepackage{epsf,epsfig}

% Bibtex and extras.
\usepackage{natbib}
\usepackage{url}
\usepackage[bookmarks=false,colorlinks=true,urlcolor={green},linkcolor={green},pdfstartview={XYZ null null 1.22}]{hyperref}

% Math.
\usepackage{amssymb}
\usepackage{amsmath}
\usepackage{mathtools}
\usepackage{cancel}
\everymath{\displaystyle}

% Exam.
\addpoints
\printanswers % \noprintanswers
\usepackage{color}
\definecolor{SolutionColor}{rgb}{0.8,0.9,1}
\shadedsolutions
\renewcommand{\solutiontitle}{\noindent\textbf{Solução:}\par\noindent}
\pagestyle{headandfoot}
\footer{}{Página \thepage\ de \numpages}{}
\boxedpoints
\pointsinrightmargin
\pointpoints{ponto}{pontos}
\hqword{Questão}
\hpword{Pontos}
\hsword{Nota}
% \qformat{\textbf{Question\thequestion}\quad(\thepoints)\hfill}

% User commands.
\newcommand{\pd}[2]{\dfrac{\partial #1}{\partial #2}}

% PDF metadata.
\pdfinfo{% hyperref overrides this
  /Title    (Prova Alternativa (Trabalho) -- Ondas e Marés)
  /Author   (Filipe Fernandes)
  /Creator  (Filipe Fernandes)
  /Producer (Filipe Fernandes)
  /Subject  (prova)
  /Keywords (prova, oceanografia)
}

% Front page.
\title{Prova Alternativa (Trabalho) -- Ondas e Marés}
\author{Prof. Filipe Fernandes}
\date{20-Dez-2013}

\begin{document}
\maketitle
\doublespacing

\begin{minipage}{.8\textwidth} % 25 pontos.
Esse Trabalho complementa a Prova Alternativa em \numpoints\ pontos.  Deve ser feito
individualmente.  Consultas ao material de aula ou referências sugeridas é
permitida.
\vspace{1cm}
\end{minipage}


\begin{questions}
\question[2\half]
Escreva as {\bf três componentes da equação do movimento} nas direções
$x, y, z$ e a {\bf equação da continuidade}.

\begin{itemize}
  \item Explique {\bf cada termo} das equações;
  \item Mostre o desenvolvimento de como chegar na equação de {\bf Laplace}
        abaixo.  Comente as {\bf aproximações} a medida que vai aplicando as
        mesmas.
\[
  \nabla^2 \tilde{p}= 0
\]
\end{itemize}

\begin{solution}
\begin{align*}
x: \pd{u}{t} + u\pd{u}{x} + v\pd{u}{y} + w\pd{u}{z} &= fv - \dfrac{1}{\rho}\pd{p}{x} + \dfrac{1}{\rho}\left(\pd{\tau^{xx}}{x} + \pd{\tau^{xy}}{y} + \pd{\tau^{xz}}{z}\right) \\
y: \pd{v}{t} + u\pd{v}{x} + v\pd{v}{y} + w\pd{v}{z} &= -fv - \dfrac{1}{\rho}\pd{p}{y} + \dfrac{1}{\rho}\left(\pd{\tau^{xy}}{x} + \pd{\tau^{yy}}{y} + \pd{\tau^{yz}}{z}\right)  \\
z: \pd{w}{t} + u\pd{w}{x} + v\pd{w}{y} + w\pd{w}{z} &= - \dfrac{1}{\rho}\pd{p}{z} -\rho g + \dfrac{1}{\rho}\left(\pd{\tau^{xz}}{x} + \pd{\tau^{yz}}{y} + \pd{\tau^{zz}}{z}\right)  \\
\pd{u}{x} + \pd{v}{y} + \pd{w}{z} &= 0
\end{align*}

Notas: Explicar termos locais, advectivos, Coriolis
\end{solution}


\question[2\half]
Leia o bloco sobre {\bf Condições de Contorno} (CC) na equação de Laplace da
Wikipédia:

\url{http://en.wikipedia.org/wiki/Laplace's_equation#Boundary_conditions}

Faça uma {\bf tradução livre} do que entende sobre os exemplos da aplicação da
CC juntamente com um {\bf paralelo} (máximo de 1 parágrafos) com as Condições de
Contorno usadas em {\bf aula}.

(Dica: leia mais sobre Condições de Contorno na Wikipédia para poder redigir a
sua resposta!)

\question[2]
Sabemos que a solução da equação de Laplace, com as nossas condições de contorno,
resulta na seguinte relação de dispersão,

\[
  \omega^2 = gk\tanh(kh)
\]

Também conhecida como relação de dispersão de ondas de gravidade de superfície.

{\bf Simplifique} a relação de dispersão usando as propriedades que você sabe
sobre a {\bf tangente hiperbólica} ($\tanh$).  Crie uma tabela com as
{\bf relações simplificadas} e as {\bf velocidades de fase} e {\bf grupo}.

Comente os {\bf dois fenômenos} de ondas de que podemos explicar com essas
formas simplificadas.

\question[1]
Abaixo vocês encontram o {\bf Modelo de Águas Rasas} (MAR) simplificado para um
fundo plano:

\begin{align}
  x\rightarrow \pd{u}{t} + u\pd{u}{x} + v\pd{u}{y} &= +fv -g\pd{\eta}{x} \label{eq:x-dir} \\
  y\rightarrow \pd{v}{t} + u\pd{v}{x} + v\pd{v}{y}&= -fu -g\pd{\eta}{y} \label{eq:y-dir} \\
  \text{Continuidade}\rightarrow\pd{\eta}{t} &= -h\left({\pd{u}{x}} + \pd{v}{y}\right) \label{eq:continuity}
\end{align}

Porque não temos as equações para a dimensão vertical $z$?

(Dica: pense na razão de aspecto de ondas muito longas!)

\question[1]
Corte os termos {\bf advectivos} e de {\bf Coriolis} do nosso sistema de equações
{\bf MAR} e combine as equações para chegar na forma:

\[
\dfrac{\partial^2\eta}{\partial x^2} - \dfrac{1}{gh}\dfrac{\partial^2\eta}{\partial{t^2}} = 0
\]

\question[1]
Mostre que a forma de onda $\eta = \eta_o \cos(kx - \omega t)$ é uma solução
para a equação acima e {\bf comente} que tipo de solução você encontrou.  Faça
uma relação entre $\omega$ e $f$ para onde essas ondas são válidas.

\question[1]
Retornando ao {\bf MAR}, corte {\bf apenas} os termos advectivos, {\bf mantendo}
os termos de Coriolis.  Faça uma {\bf nova} relação entre $\omega$ e $f$ para
onde essa aproximação é válida.

\question[4]
Comente sobre a diferença entre $f$ e $f_o$ e onde usamos um ou outro ou ambos
para resolver as ondas influenciadas pela rotação da Terra abaixo:
\begin{itemize}
  \item Ondas de Poincaré
  \item Ondas de Kelvin Costeiras
  \item Ondas de Kelvin Equatoriais
  \item Ondas de Rossby
\end{itemize}

Combine as equações do {\bf MAR} para chegar nas equações de {\bf Kelvin},
{\bf Poincaré} e {\bf Rossby} (respectivamente) abaixo:

(Dica: cheque os slides de aula para as aproximações de cada solução!)

\begin{align*}
  \dfrac{\partial^2\eta}{\partial y^2} - \dfrac{1}{gh}\dfrac{\partial^2\eta}{\partial{t^2}} &= 0 \\
  \dfrac{\partial^2\eta}{\partial x^2} - \dfrac{1}{gh}\dfrac{\partial^2\eta}{\partial t^2} &= f_o^2\eta \\
  \dfrac{\partial{}}{\partial t} \nabla^2 \eta - \dfrac{f_o^2}{gh} \eta + \beta \dfrac{\partial{\eta}}{\partial x} &= 0
\end{align*}

\question[5]
Discorra (escreva de 2 a 3 parágrafos) sobre os {\bf temas}:
\begin{itemize}
  \item[a)] A natureza da ondas de Kelvin, de ser uma onda em uma direção e um
        balanço geostrófico em outra, resulta nos pontos anfidrômicos nos
        oceanos.
  \item[b)] Ondas de Rossby e o clima estão intimamente relacionados.
  \item[c)] Ondas de Rossby no oceano têm um papel fundamental na formação das
        correntes de contorno Oeste.
  \item[d)] Fale sobre: Maré de equilíbrio e a manifestação da maré na
            plataforma costeira.
  \item[e)] Como utilizamos a teoria de Fourier para analisar marés no oceano?
\end{itemize}

\question[4]
Uma forma de oscilações que não comentamos nesse curso são os {\bf Movimentos
Inerciais} no oceano.  (Isso porque tal tema é abordados em Oceanografia Física
Descritiva e/ou Dinâmica dos Oceanos.)

Comente {\bf o que são} os movimentos inerciais, {\bf como eles acontecem}
no oceano e explique a animação vista nesse site:

\url{http://www.incois.gov.in/Tutor/IntroOc/por/notes/figures/fig6a1.html}

Adicionalmente, vá ao site:

\url{http://oceanmotion.org/html/resources/coriolis.htm}

Crie 3 figuras clicando em altas ($\pm$ 70\textdegree{}), baixas
($\pm$ 5\textdegree{}) e médias latitudes ($\pm$ 45\textdegree{}).

{\bf Comente} sobre os {\bf tamanhos} dos círculos inerciais em cada uma delas.

{\bf Calcule} as frequências inerciais para as latitudes de Santos-SP, Salvador-BA e
Rio Grande-RS.  Inverta as frequências para períodos lembrando que
$Ti = \dfrac{2\pi}{f}$.  Comente sobre a proximidade e/ou distância desses
períodos para com os períodos das marés M$_2$, O$_1$.

\question[1]
Comente {\bf detalhadamente} (uma redação de 1 página), sobre as
{\bf dificuldades} que teve no curso (tópicos de matemática, didática do
professor, língua estrangeira, obtenção de bibliografia e material etc).
Elabore um plano {\bf realístico} de {\bf superação} dos problemas que estão
ao seu alcance e faça sugestões aos que não estão diretamente ao seu alcance.

\end{questions}
\end{document}
