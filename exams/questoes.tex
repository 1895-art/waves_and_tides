Espectro de ondas
-----------------
O espectro de ondas abaixo é típico de uma estação de obervação costeira.
Explique em detalhes a provável razão da existência dos 2 picos proeminentes no
sinal.

Ondas de gravidade de superfície Longa vs. Curta
------------------------------------------------
Se a profundidade média do canal de Santos é 14 m, use os limites de
$\dfrac{h}{\lambda}$ para calcular o comprimento de onda, período e velocidades
de fase e grupo para:
a. A onda mais curta ``rasa'' (ou ``longa'');
b. A onda mais longa ``profunda'' (ou ``curta'');
c. Calcule as mesmas informações para a plataforma com uma profundidade média de
200 m.

Ressonância de onda de 1/4
--------------------------
Considere o caso onde a forçante de oceano aberto na boca do canal é devido a
maré M2 (principal lunar semi-diurna).  Calcule o comprimento de onda
($\lambda = \dfrac{2\pi}{k}$) para uma onda estacionária em um canal nas
seguintes profundidades:

* ho = 1 m
* ho = 10 m
* ho = 100 m
* ho = 4000 m

Em cada caso calcule o comprimento da bacia que levaria e 1/4 de ressonância na
onda.  Identifique uma feição oceânica real que corresponda a cada profundidade
do canal e largura da bacia.

Tsunamis em oceano profundo
---------------------------
Um terremoto na costa do Peru cria uma onda com comprimento $\lambda >> 4$ km.
Estime o tempo que leva para essa onda atingir o Japão.


Relação de dispersão de ondas de gravidade
------------------------------------------
Começando com a equação de ondas de gravidade para águas rasas (ou ondas longas),
sem rotação nem bordas costeiras):

\[
  \dfrac{\partial^2 \eta}{\partial x^2}  - \dfrac{1}{gh}\dfrac{\partial^2 \eta}{\partial t^2} = 0
\]

a.) Mostre que essa equação pode ser satisfeita pela forma de onda:
\[
  \eta = \eta_o \cos(kx - \omega t)
\]

b.) De acordo com os seus cálculos na parte acima qual é o constrição desse tipo
de ondas que encontramos usando esse sistema de equações?
(Dica:  Você deve reconhecer pela relação de dispersão de ondas de água rasa).


\question
Identifique a {\bf amplitude}, o {\bf comprimento}, a {\bf frequência},
o {\bf período} e o {\bf número de onda} nas formas abaixo.

\begin{parts}
  \part[3]
  $ \eta = 10 \cos(4x + 2t) $

  \begin{solution}
    a = 10, k = 4, L = $\frac{2\pi}{k} = \frac{\pi}{2}$, $\omega$ = −2,
    T = $\frac{2\pi}{\omega} = −\pi$
  \end{solution}

  \part[3]
  $ \eta = −0.5 \sin(\pi x − t) $

  (Ponto extra para quem re-escrever a segunda onda utilizando a função cosseno.)

  \begin{solution}
  a = −0.5, k = $\pi$, L = $\frac{2\pi}{k}$ = 2, $\omega$ = 1,
  T = $\frac{2\pi}{\omega} = 2\pi$

  {\bf Ponto extra:}
  Essa onda pode ser re-escrita na forma,  $\eta = −0.5 \cos(\pi x − t + \pi/2)$
  \end{solution}
\end{parts}


\question
Dada a relação de dispersão de ondas de gravidade,

\[
  \omega^2 = gk\tanh(kh),
\]


e as seguintes informações sobre a tangente hiperbólica e a relação entre
profundidade $h$ e comprimento de onda $L$:

{\bf Ondas longas (ou ondas de águas rasa):}
\begin{itemize}
  \item $\tanh(kh) \sim kh; h << L$
  \item $\frac{h}{L} < \frac{1}{20}$
\end{itemize}

{\bf Ondas curtas (ou ondas de água profunda):}
\begin{itemize}
  \item $\tanh(kh) \sim 1; h >> L$
  \item $\frac{h}{L} > \frac{1}{2}$
\end{itemize}

Responda:

\begin{parts}
  \part[4] Faça a aproximação para águas rasa (ondas longas) e águas profundas
  (ondas curtas) na {\bf relação de dispersão} e crie uma tabela com as equações
  para $C$ e $Cg$ em cada um dos regimes de ondas.

  \begin{solution}
    \raggedright
    {\bf Onda longa (água rasa):}\\
    Dispersão: $\omega = k\sqrt{gh}$\\
    Vel. de Fase: $C = \dfrac{\lambda}{T} = \dfrac{\omega}{k} = \sqrt{gh}$\\
    Vel. de Grupo: $Cg = \pd{}{k}\left(\omega = k\sqrt{gH}\right)
    \rightarrow C_g = \sqrt{gh}$\\

    {\bf Ondas curtas (água profunda):}\\
    Dispersão: $\omega = \sqrt{gk}$\\
    Vel. de Fase: $C = \dfrac{\lambda}{T} = \dfrac{\omega}{k} = \sqrt{g/k}$\\
    Vel. de Grupo: $C_g = \pd{}{k}\left[\omega = g^{1/2}k^{1/2}\right]
    \rightarrow C_g = \dfrac{1}{2}g^{1/2}k^{-1/2} = \dfrac{C}{2}$\\
  \end{solution}

  \part[2]
  Explique o fenômeno de {\bf dispersão} e {\bf refração} de ondas usando a
  {\bf velocidade de fase} ($C$), a {\bf velocidade de grupo} ($C_g$) e a
  {\bf profundidade} (h).

  \begin{solution}
    {\bf Dispersão:} Ocorre devido a velocidade de fase ($C$) de ondas curtas
    ser o dobro da velocidade de grupo ($C_g$), assim algumas ondas ``escapam''
    do grupo levando a seleção das ondas.

    {\bf Refração:} A refração ocorre sempre que uma parte da onda longa estiver
    em profundidades diferentes ($h$), assim partes diferentes da mesma onda
    terão velocidades de fase (e de grupo $C=C_g=\sqrt{gh}$) diferente forçando
    a mesma a girar.
  \end{solution}

  \part[2]
  Para chegar na relação de dispersão e na solução de águas profundas fizemos
  várias {\bf aproximações}.  Cite {\bf 3} aproximações e explique sua validade
  {\bf física}, onde/porque ela é {\bf aceitável} e quando (se algum momento)
  ela pode ser {\bf invalidada}.

  \begin{solution}
    \begin{itemize}
      \item Longe o sítio da forçante do vento
      \item Longe da zona de arrebentação e quebra;
      \item Dimensões horizontais infinitas;
      \item Sem atrito;
      \item Período da onda muito menor que o período inercial;
      \item Assume-se um estado médio e perturbações sobre esse (as ondas);
      \item A amplitude da onda é pequena quando comparada com a coluna d'água.
    \end{itemize}
  \end{solution}

  \part[2]
  Quando assumimos uma forma de onda para a solução da
  {\bf amplitude de pressão} $\mathbb{P}$,

  \begin{equation}
    \mathbb{P}(z) = \cos(\theta),
    \label{eq:P}
  \end{equation}

  na equação de Laplace,

  \begin{equation}
    \nabla^2 \tilde{p} = 0,
    \label{eq:Laplace}
  \end{equation}

  resultamos na Equação Diferencial de Segunda Ordem Homogênea abaixo:

  \begin{equation}
    \frac{\partial^2\mathbb{P}}{\partial z^2} - \mathbf{K}^2\mathbb{P}=0
    \label{eq:dif}
  \end{equation}

  Para resolvermos essa equação precisamos de duas condições de contorno.
  Explique:

  \begin{itemize}
    \item Por que {\bf não} utilizamos condições de contorno em todas as
          dimensões $x$, $y$, $z$? Onde colocamos as condições de contorno?
          {\bf Explique a sua resposta!}
    \item Ambas condições de contorno são {\bf dinâmicas}, {\bf ou estáticas}?
          Se for(em) dinâmica(s), como fazemos para ``{\bf acompanhar}'' a
          variável enquanto ela muda?
  \end{itemize}

  (Ponto extra: Substitua a equação \ref{eq:P} em \ref{eq:Laplace} e chegue na
  diferencial \ref{eq:dif} lembrando que $\mathbf{K} = \sqrt{k^2 + l^2}$.)

  \begin{solution}
    1) Utilizamos condições de contorno apenas na {\bf dimensões vertical}
    ($z$), deixando as dimensões horizontais ($x$, $y$) como ``infinitas''.
    Isso porque as ondas se manifestam como {\bf alterações verticais na
    pressão.}

    2) {\bf Não}.  Na superfície usamos a condição {\bf dinâmica} com
    continuidade de pressões.  Já fundo usamos condição {\bf estática} sem
    movimento ``cruzando'' o fundo).

    3) Na condições de contorno dinâmica usamos uma expansão em {\bf Série de
    Taylor} para centrar a condição de contorno em $z=0$.
  \end{solution}
\end{parts}

\question
Cálculos básicos de ondas.

\begin{parts}
  \part[2]
  Se {\bf 16 cristas} de ondas passam {\bf sucessivamente} por um ponto fixo num
  intervalo de {\bf 1 minuto e 40 segundos}, qual é a {\bf frequência angular}
  dessas ondas?

  \begin{solution}
    $T = \dfrac{60 + 40}{16} = 6.25$ s

    $\omega = \dfrac{2\pi}{6.25} \sim 1$ s$^{-1}$
  \end{solution}

  \part[2]
  O {\bf período} de uma onda é de {\bf 25 segundos}.  Qual seria a
  {\bf velocidade} dessa onda em {\bf águas profundas}?

  \begin{solution}
  \begin{align*}
    \omega &= \sqrt{gk} \therefore k = \omega^2/g, \\
    C &= \sqrt{g/k} \therefore C = \sqrt{g^2/\omega^2} = \dfrac{g}{\omega} \\
    \omega &= \dfrac{2\pi}{25} = 0.2513 \text{ s}^{-1} \\
    C &= \dfrac{9.8}{0.2513} = 38.99 \text{ m s}^{-1} \\
  \end{align*}
  \end{solution}

  \part[2]
  Qual seria a {\bf velocidade} de uma onda com {\bf comprimento} de ondas
  de 312 m em {\bf águas profundas}?  E em {\bf águas rasas}?

  \begin{solution}
    $C = \sqrt{g/k} = \sqrt{9.8 / 0.02013} = 22.06$ m s$^{-1}$

    Já para águas rasas (ondas longas) temos como calcular o limite superior
    das velocidades.

    $\dfrac{h}{312} < \dfrac{1}{20} \therefore h <  15.6$ m

    $C < \sqrt{9.8 \times 15.6}$

    $C < 12.36$ m s$^{-1}$
  \end{solution}
\end{parts}

\question[3]
Sabemos que a geração de ondas de superficiais de gravidade está associada ao
vento.  Cite os {\bf 3 fatores} que precisamos saber sobre o {\bf vento} para
estimar a quantidade de energia que será transferida para as ondas geradas.

(Ponto extra: Há algum fator {\bf limitante} que não está associado ao vento?)

\begin{solution}
  \begin{itemize}
    \item A velocidade do vento;
    \item Pista do vento (a distância em que o vento sopra);
    \item Duração do vento;
  \end{itemize}
  {\bf Ponto extra:} Profundidade da coluna d'água é um fator limitante
  (ressacas costeiras).
\end{solution}

\end{questions}


Ondas internas vs. Ondas Externas
---------------------------------
Q: Ondas externas são oscilações entre a interface ar-mar, já as ondas internas são
oscilações na interface entre densidades.  Sabendo que quanto maior a diferença
de densidade mais energia é necessário para mudar o estado do sistema, onde as
oscilações são mais facilmente estabelecidas? Em uma interface interna ou na
superfície do mar?

A: As oscilações são mais facilmente estabelecidas em uma interface interna do
que na superfície do mar.  A diferença de densidade entre duas camadas de água é
menor do que entre água e ar.  Densidade do ar = 1,2 kg m$^{-3}$ e densidade da
água do mar = 1025 kg m$^{-3}$.  Consequentemente menos energia ́e requerida para
gerar ondas internas do que ondas de superfície de similar amplitude.


Q: Quais ondas, de amplitude similar, ``viajam'' mais lentamente: ondas de
superfície ou ondas internas?

A: Ondas internas ``viajam'' mais lentamente que ondas de superfície de similar amplitude.

Q: Quais ondas são mais importantes no contexto de processos de mistura vertical:
ondas de superfície ou ondas internas?

A: Ondas internas s ao mais importantes no contexto de processos de mistura
vertical, especialmente quando quebram.


\question
O diagrama \ref{fig:shallow_water} representa a geometria do modelo de águas
rasas (equações \ref{eq:x-dir}, \ref{eq:y-dir} e \ref{eq:continuity}).

\begin{figure}[ht]
  \centering
  \includegraphics[scale=1]{../../figures/aguas_rasas.pdf}
  \label{fig:shallow_water}
\end{figure}


\begin{align}
  \pd{u}{t} + u\pd{u}{x} + v\pd{u}{y} &= +fv -g\pd{\eta}{x} \label{eq:x-dir} \\
  \pd{v}{t} + u\pd{v}{x} + v\pd{v}{y}&= -fu -g\pd{\eta}{y} \label{eq:y-dir} \\
  \pd{\eta}{t} + \pd{}{x}(Hu) + \pd{}{y}(Hv) &= 0 \label{eq:continuity}
\end{align}

 \begin{parts}
  \part[2]
  Explique com suas palavras $\eta$, $h$, $H$ e $b$.

  \begin{solution}
    $\eta \rightarrow$ Elevação da superfície livre.  Varia no espaço-tempo.\\
    $h \rightarrow$ Altura média da coluna d'água.  Não varia no espaço-tempo.\\
    $H \rightarrow$ Altura da coluna d'água.  Varia no espaço-tempo (Com $\eta$  e $b$).\\
    $b \rightarrow$ Perfil batimétrico.  Varia apenas no espaço.
  \end{solution}

  \part[3]
  Em aula assumimos que o número de Rossby ($R_o$) e muito menor que 1.
  \[ R_o = \frac{U}{fL} << 1 \]

  Sabendo que $R_o$ é a escala da {\bf velocidade} sobre a escala {\bf horizontal
  de comprimento} divididos pelo {\bf parâmetro de Coriolis} quais expressões das
  equações \ref{eq:x-dir} e \ref{eq:y-dir} podemos eliminar?  A equação fica
  completamente linear?

  \begin{solution}
  \( R_o = \frac{U}{fL} << 1 \) significa que podemos cortar todos os termos
  advectivos.  Pois a escala da velocidade $U (u,v,uu,uv,vv)$ sobre a escala de
  tamanho $L (x,y)$ é muito menor que a frequência de Coriolis $f$
  \( \frac{U}{L} << f\).

  \begin{align*}
  \pd{u}{t} + \cancel{u\pd{u}{x}} + \cancel{v\pd{u}{y}} &= +fv -g\pd{\eta}{x} \\
  \pd{v}{t} + \cancel{u\pd{v}{x}} + \cancel{v\pd{v}{y}} &= -fu -g\pd{\eta}{y}
  \end{align*}
  \end{solution}

  \part[3]
  Assumindo $R_o << 1$ nos permitiu eliminar os termos não lineares das equações
  do movimento nas direções $x$ e $y$ (\ref{eq:x-dir} e \ref{eq:y-dir}).  Porém,
  a equação da continuidade (\ref{eq:continuity}) ainda é não-linear.

  Identifique o termo não linear e explique como aproximação abaixo nos
  permite linearizar o problema.

  (Dica: A primeira aproximação é chamada de ``ondas de pequena amplitude''.  Já
  a segunda nos diz algo sobre a batimetria!)

  \begin{align*}
  \eta &<< H\\
  b &<< h
  \end{align*}

  \begin{solution}
   A aproximação de ondas de pequena amplitude ($\eta << H$) diz que a elevação
   do nível do mar é muito menor que a coluna d'água.  Isso nos permite dizer
   que $H$ não varia no tempo com $\eta$.

   A aproximação feita com a batimetria ($b<<h$) nos diz que a variação
   batimétrica é muito menor que á média da coluna d'água.  Isso nos permite
   ignorar as variações da topografia de fundo, logo $h \sim H$.  Agora podemos
   simplificar a equação da continuidade para a forma linear assumindo um $h$
   constante:
   \[ \pd{\eta}{t} = -h\left( \pd{u}{x} + \pd{v}{y}\right)\]
  \end{solution}

  \part[2] Nós também assumimos o número de Rossby temporal ($R_{oT}$) como
  sendo aproximadamente 1:
  \[ R_{oT} = \frac{1}{fT} \sim 1 \]

  O que essa aproximação significa?  Como era o Rossby temporal das ondas curtas
  e longas não influenciadas pela rotação da Terra?

  \begin{solution}
   Quando o número de Rossby temporal é aproximadamente ``1'' a rotação da
   Terra se torna importante na escala em estudo pois a frequência do movimento
   $\frac{1}{T}$ é aproximadamente igual à frequência de Coriolis $f$.
   \[ \frac{1}{T} \sim f \]

   Nas ondas longas e curtas não influenciadas pela rotação da Terra nos temos
   que a frequência do movimento é muito maior que a frequência de Coriolis.
   \( \frac{1}{T} >> f \), ou \( R_{oT} >> 1 \)
  \end{solution}
 \end{parts}

\question
 Em aula nos vimos ondas 3 tipos de ondas influenciadas pela rotação da Terra:
 ondas de Poincaré, Kelvin e Rossby.  Todas derivadas do nosso
 modelo de águas rasas.  Porém, cada uma assume diferente premissas para
 simplificar o problema e resolver a onda em questão.

\begin{parts}
 \part[2]
 Como as ondas de Kelvin e Poincaré se diferenciam das ondas de Rossby?\\
 (Dica: $f = f_o + \beta y$)

 \begin{solution}
  Ondas de Poincaré e Kelvin são resolvidas no plano-$f$ ($f_o$) ou seja, o
  efeito da rotação é simplificado para primeira ordem como sendo uma expansão
  de Taylor ao redor da latitude.
  \[f \sim f_o = 2\Omega\sin\phi_o \]
  Isso ignora os efeitos da esfericidade da Terra (representada pelo
  plano-$\beta$), essencial para a existências das ondas de Rossby.
 \end{solution}

  \part[1]
  A onda de Kelvin é um caso especial de onda de Poincaré que viaja sempre com
  um contorno do seu lado esquerdo (direito) no hemisfério sul (norte).  Dito
  isso, simplifique as equações do nosso modelo de águas rasas (\ref{eq:x-dir},
  \ref{eq:y-dir} e \ref{eq:continuity}) para uma ondas de Kelvin.

\vspace{2cm}
\begin{solution}
  \begin{align*}
  \cancelto{0}{\pd{u}{t}} &= +fv -g\pd{\eta}{x}\\
  \pd{v}{t} &= \cancelto{0}{-fu} -g\pd{\eta}{y}\\
  \pd{\eta}{t} &= -h\left(\cancelto{0}{\pd{u}{x}} + \pd{v}{y}\right)
  \end{align*}
\end{solution}

  \part[3]
  A relação de dispersão das ondas de Poincaré é:
  \[\omega^2 = f_o^2 + ghk^2\]

  Discuta os extremos onde:\\
  $\omega >> f_o$ e $\omega \sim f_o$\\
  (Dica, prove que $k=0$ quando $\omega \sim f_o$)

\begin{solution}
  $\omega >> f_o \rightarrow$ Temos de volta a relação de dispersão de ondas de água
  rasa.\\
  \[\frac{\omega}{k} = \sqrt{gh} \]
  $\omega \sim f_o \rightarrow 0 = ghk^2$, como sabemos que $g$ e $h$ não são zero,
  temos $k=0$
\end{solution}

  \part[2]
  A solução das velocidades $u$ e $v$ e $\eta$ de ondas de Poincaré é:
  \begin{align*}
    u &= u_o\cos(kx - \omega t)\\
    v &= -\frac{f_o}{\omega}u_o\sin(kx - \omega t)\\
    \eta &= \frac{khu_o}{\omega}\cos(kx - \omega t)
  \end{align*}

Podemos ver que a componente zonal ($x$) oscila como uma onda de águas rasa,
enquanto a componente meridional ($y$) oscila ortogonalmente à $x$ e
é ``escalada'' pelo termo $\frac{f_o}{\omega}$.  Também sabemos que no extremo
onde $\omega = f_o$, $k=0$. Como ficam as oscilações descritas acima?

(Dica: Essas são as oscilações inerciais, ou seja, oscilações com a frequências
de Coriolis $f$.)

\begin{solution}
Sem elevação $\eta = 0$, $u$ e $v$ oscilam como ``oscilações inerciais''.
\begin{align*}
  u &= u_o\cos(\cancelto{0}{{kx}} - \omega t)\\
  v &= -u_o\sin(\cancelto{0}{{kx}} - \omega t)\\
  \eta &= \cancelto{0}{\frac{khu_o}{\omega}\cos(kx - \omega t)}
\end{align*}
\end{solution}

  \part[2]
  A onda de Kelvin, como já vimos acima, precisa de um ``contorno'' para se
  propagar.  Em geral, essa ondas são a manifestação das marés na plataforma
  costeira.  Porém, temos um caso especial de ondas de Kelvin chamada de ondas
  de Kelvin Equatorial.  Essas ondas são responsáveis pelo fenômeno conhecido
  como {\it el-niño}.

  Explique como esse tipo de onda pode existir.  Use
  $f = f_o + \beta y$ em sua explicação.\\
  (Dica: Como é a forma de $f_o$ no equador?)

  \begin{solution}
   Como $f = f_o + \beta y$, apesar de $f_o = 0$, $f$ não é zero, pois ainda tem
   o efeito $\beta$.  Já o ``contorno'' necessário para sua existência é a forte
   variação de $f$ ao redor do Equador, pois sai de próximo a zero para um valor
   significativo ($\pm 5^{\circ}$ graus de latitude).
  \end{solution}

\end{parts}

\question
Ondas de vorticidade (ou ondas Rossby).

Para descrever as ondas de Rossby em aula nós utilizamos a forma completa de $f$.

\begin{align*}
  \pd{u}{t} &= (f_o + \beta y)v - g\pd{\eta}{x}\\
  \pd{v}{t} &= -(f_o + \beta y)u - g\pd{\eta}{y}\\
  \pd{\eta}{t} &= -h\left(\pd{u}{x} + \pd{v}{y}\right)
\end{align*}

\begin{parts}
\part[3]
Explique o que a aproximação do plano $\beta$ representa.

\begin{solution}
 O plano $\beta$ é a aproximação para representar a esfericidade da Terra em
 coordenadas cartesianas.
 \[ \beta = \frac{2\Omega\cos(\phi_o)}{R}\]
\end{solution}

\part[1]
A relação de dispersão de ondas de Rossby é dada por:
\[\omega = \frac{-\beta k}{k^2 + l^2 + \frac{f_o^2}{gh}}\]

Prove que essas ondas se propagam energia apenas para Oeste.

(Dicas: Energia se propaga com a velocidade de fase $C$.)

\begin{solution}
 \[
    C \equiv \frac{\omega}{k} = \frac{-\beta}{k^2 + l^2 + \frac{f_o^2}{gh}}
 \]
 $C$ é sempre negativo, ou seja se propaga na direção $-x$, Oeste no nosso
 sistema de coordenadas.
\end{solution}


\part[1]
Sabemos que as correntes oceânicas são intensificadas no contorno oeste.  Qual é
o papel das ondas de Rossby nesse efeito?  Sabemos que as ondas de Rossby
também ocorrendo na atmosfera, porque não temos a intensificação na atmosfera?

\begin{solution}
 As ondas transportam energia.  Como as ondas de Rossby viajam apenas para Oeste,
 a sua energia fica acumulada nos contornos dos continentes.  Já na atmosfera
 não há contornos, logo não á intensificação.
\end{solution}

\end{parts}
