% Style.
\documentclass[letterpaper,portuguese,12pt,pdftex]{exam}

\usepackage{setspace}
% \usepackage{lineno}
\usepackage[left=2cm,top=2cm,right=2.5cm, bottom=2cm]{geometry}
% \usepackage[left=2.5cm,top=3cm,right=2.5cm]{geometry}

% Portuguese.
\usepackage[brazil]{babel}
\usepackage[T1]{fontenc}
\usepackage[utf8x]{inputenc}
\usepackage{textcomp}

% Font.
\usepackage{lmodern}

% Figures.
\usepackage{epsf,epsfig}

% Bibtex and extras.
\usepackage{natbib}
\usepackage{url}
\usepackage[bookmarks=false,colorlinks=true,urlcolor={green},linkcolor={green},pdfstartview={XYZ null null 1.22}]{hyperref}

% Math.
\usepackage{amssymb}
\usepackage{amsmath}
\usepackage{mathtools}
\usepackage{cancel}
\everymath{\displaystyle}

% Exam.
\addpoints
\printanswers % \noprintanswers
\usepackage{color}
\definecolor{SolutionColor}{rgb}{0.8,0.9,1}
\shadedsolutions
\renewcommand{\solutiontitle}{\noindent\textbf{Solução:}\par\noindent}
\pagestyle{headandfoot}
\footer{}{Página \thepage\ de \numpages}{}
\boxedpoints
\pointsinrightmargin
\pointpoints{ponto}{pontos}
\hqword{Questão}
\hpword{Pontos}
\hsword{Nota}
% \qformat{\textbf{Question\thequestion}\quad(\thepoints)\hfill}

% User commands.
\newcommand{\pd}[2]{\dfrac{\partial #1}{\partial #2}}

% PDF metadata.
\pdfinfo{% hyperref overrides this
  /Title    (Prova Alternativa -- Ondas e Marés)
  /Author   (Filipe Fernandes)
  /Creator  (Filipe Fernandes)
  /Producer (Filipe Fernandes)
  /Subject  (prova)
  /Keywords (prova, oceanografia)
}

% Front page.
\title{Prova Alternativa -- Ondas e Marés}
\author{Prof. Filipe Fernandes}
\date{20-Dez-2013}

\begin{document}
\maketitle
\doublespacing

\vspace{2cm}
\hbox to \textwidth{Nome e número de matrícula:\enspace\hrulefill}
\vspace{2cm}

\begin{minipage}{.8\textwidth}
Esse exame incluí \numquestions\ questões. O número total de pontos é \numpoints.
\vspace{1cm}

Essa prova segue o Acordo Ortográfico da Língua Portuguesa de 1990 (em vigor no
início de 2009).  Por isso erros de ortografia e gramática serão descontados da
sua nota final.

\vspace{1cm}

A prova deve ser feita individualmente e sem consulta.  O aluno deverá usar
CANETA (preta ou azul) para responder as questões – qualquer questão respondida
à lápis não será considerada na hora da correção.  Coloque seu nome em TODAS as
folhas e numere as mesma colocando o número total de folhas (Ex.: 1/4, 2/4 e
etc).

\vspace{1cm}

Leia atentamente todas as questões: a interpretação faz parte da prova e dúvidas
serão esclarecidas apenas após o término da mesma.

\end{minipage}

\clearpage
% 10


\begin{questions}
\question[2]
  Sabendo que equação de onda é um senoide do tipo $\eta = \cos(kx -\omega t)$.
  Defina e escreva a formulação matemática para {\bf número de onda} $k$ e
  {\bf frequência angular} $\omega$ mostrando a sua relação com {\bf período}
  $T$ e {\bf comprimento} de onda $L$.  Comente sobre as suas unidades e seu
  significado.

  \begin{solution}
    \begin{align*}
      \omega &= \dfrac{2\pi}{T} \text{ (s}^{-1}) \text{: inverso do período (s).}\\
      k &= \dfrac{2\pi}{\lambda {\text{ (ou L)}}} \text{ (m}^{-1}) \text{: inverso do comprimento (m).}
    \end{align*}
  \end{solution}

\question
Identifique a {\bf amplitude}, o {\bf comprimento}, a {\bf frequência}, o
{\bf período} e o {\bf número de onda} nas formas abaixo.

\begin{parts}
  \part[1]
    \[
      \eta = 10 \cos(4x - 2t)
    \]

    \begin{solution}
      A = 10, k = 4, L = $\frac{\pi}{2}$, $\omega$ = 2,
      T = $\pi$
    \end{solution}

    \part[1]
    \[
      \eta = 0.5 \cos(\pi x - t)
    \]

    \begin{solution}
      A = 0.5, k = $\pi$, L = 2, $\omega$ = 1,
      T = $2\pi$
    \end{solution}
\end{parts}

\question[2]
Sabemos que processos de {\bf mistura vertical} são cruciais para a vida
marinha, tornando nutrientes mais acessíveis a biota.  Discorra sobre o papel
das ondas internas na mistura vertical.  (Dica: Lembre-se que o oceano pode ser
divido em 3 extratos: camada de mistura, picnoclina e camada profunda.  Nesse
contexto onde estão plankton, as ondas internas e os nutrientes?)

\begin{solution}
  As ondas internas estão geralmente em zonas de alto gradiente de densidade
  (interface interna) como a picnoclina.  Quando há ondas internas nessa região,
  seu movimento vertical pode disponibilizar nutrientes da camada profunda logo
  abaixo para a camada de mistura acima.  Isso ocorre principalmente quando há
  quebra da onda interna, promovendo uma mistura entres essas interfaces.
  Disponibilizando assim nutrientes que antes estavam inacessíveis ao plankton da camada de mistura.
\end{solution}


\question[2]
Você foi contrato pela empresa ACME\circledR{} para executar uma análise de
marés na costa da praia de {\it Lost}.  Sabendo que o número de forma da maré é uma
razão entre as principais constituintes {\bf diurnas} e {\bf semi-diurnas}.
Calcule o {\bf Número de Forma} (F) para essa praia dados as amplitudes abaixo
(em metros) e comente sobre a predominância diurna ou semi-diurna no local.

\begin{center}
\begin{tabular}{ll}
Constituinte & Amplitude (m)\\
\hline
K1 & 2,4\\
O1 & 1,2\\
M2 & 0,7\\
S2 & 0,2 \\
\hline
\end{tabular}
\end{center}


\begin{solution}
  \begin{align*}
    \text{F} &= \dfrac{\text{K1 + O1}}{\text{M2 + S2}} \\
    \quad \\
    \text{F} &= \dfrac{2,4 + 1,2}{0,7 + 0,2} = 4
  \end{align*}

  Maré predominantemente diurna.
\end{solution}


\question[1]
O sistema de equações abaixo representa o modelo de águas
rasas (equações \ref{eq:x-dir}, \ref{eq:y-dir} e \ref{eq:continuity}), de onde
derivamos as relações de dispersão das ondas de Poincaré, Kelvin e Rossby.

\begin{align}
  \pd{u}{t} + u\pd{u}{x} + v\pd{u}{y} &= +fv -g\pd{\eta}{x} \label{eq:x-dir} \\
  \pd{v}{t} + u\pd{v}{x} + v\pd{v}{y}&= -fu -g\pd{\eta}{y} \label{eq:y-dir} \\
  \pd{\eta}{t} &= -h\left({\pd{u}{x}} + \pd{v}{y}\right) \label{eq:continuity}
\end{align}

Sabendo que a onda de Kelvin é um caso especial de onda de Poincaré que viaja
sempre com um contorno do seu lado esquerdo (direito) no hemisfério sul (norte).
Dito isso, simplifique as equações do nosso modelo de águas rasas
(\ref{eq:x-dir}, \ref{eq:y-dir} e \ref{eq:continuity}) para uma ondas de Kelvin.
(Você tem apenas que cortar alguns termos de cada equação e justificar o porquê
do corte!)

\begin{solution}
  \begin{align*}
  \cancelto{0}{\pd{u}{t}} &= +fv -g\pd{\eta}{x}\\
  \pd{v}{t} &= \cancelto{0}{-fu} -g\pd{\eta}{y}\\
  \pd{\eta}{t} &= -h\left(\cancelto{0}{\pd{u}{x}} + \pd{v}{y}\right)
  \end{align*}

Eliminamos os termos advectivos porque nossa ondas são lineares e adicionalmente,
fazemos a velocidade em $x$ ($u$) ser próxima de zero para garantir o equilíbrio
dessa onda no contorno.
\end{solution}

\question[1]
Ondas de vorticidade (ou ondas Rossby) tem a relação de dispersão definida
(já simplificada) por:

  \[
    \omega = -\beta k \left( \dfrac{L^2 R^2}{R^2 + L^2} \right)
  \]


Faça as considerações para {\bf Ondas de Rossby Curtas} ($L << R$, comprimento
de onda menor que raio de deformação) e {\bf Ondas de Rossby Longas} ($L >> R$,
comprimento de onda maior que o raio de deformação).

  \begin{solution}
  Onda curta (quando $L << R$),
  \begin{align*}
    \omega &= -\beta k \left( \dfrac{L^2 R^2}{R^2 + \cancel{L^2}} \right) \\
    \omega &= -\beta k \left( \dfrac{L^2 \cancel{R^2}}{\cancel{R^2}} \right) \\
    \omega &\approx -\beta k L^2
  \end{align*}

  Onda longa (quando $L >> R$),
  \begin{align*}
    \omega &= -\beta k \left( \dfrac{L^2 R^2}{\cancel{R^2} + L^2} \right) \\
    \omega &= -\beta k \left( \dfrac{\cancel{L^2} R^2}{\cancel{L^2}} \right) \\
    \omega &\approx -\beta k R^2
  \end{align*}
\end{solution}

\end{questions}
\end{document}
