% Style.
\documentclass[letterpaper,portuguese,12pt,pdftex]{exam}

\usepackage{setspace}
\usepackage{lineno}
\usepackage[left=2.5cm,top=3cm,right=2.5cm]{geometry}

% Portuguese.
\usepackage[brazil]{babel}
\usepackage[T1]{fontenc}
\usepackage[utf8x]{inputenc}
\usepackage{textcomp}

% Font.
\usepackage{lmodern}

% Figures.
\usepackage{epsf,epsfig}

% Bibtex and extras.
\usepackage{natbib}
\usepackage{url}
\usepackage[bookmarks=false,colorlinks=true,urlcolor={green},linkcolor={green},pdfstartview={XYZ null null 1.22}]{hyperref}

% Math.
\usepackage{amssymb,amsmath}
\usepackage{mathtools}
\everymath{\displaystyle}

% Exam.
\addpoints
\printanswers
% \noprintanswers
\usepackage{color}
\definecolor{SolutionColor}{rgb}{0.8,0.9,1}
\shadedsolutions
\renewcommand{\solutiontitle}{\noindent\textbf{Solução:}\par\noindent}
\pagestyle{headandfoot}
\footer{}{Página \thepage\ de \numpages}{}
\boxedpoints
\pointsinrightmargin
\pointpoints{ponto}{pontos}
\hqword{Questão}
\hpword{Pontos}
\hsword{Nota}
% \qformat{\textbf{Question\thequestion}\quad(\thepoints)\hfill}

% User commands.
\newcommand{\pd}[2]{\dfrac{\partial #1}{\partial #2}}

% PDF metadata.
\pdfinfo{% hyperref overrides this
  /Title    (Dever de casa 04 -- Ondas e Marés)
  /Author   (Filipe Fernandes)
  /Creator  (Filipe Fernandes)
  /Producer (Filipe Fernandes)
  /Subject  (HW)
  /Keywords (HW, oceanografia)
}

% Front page.
\title{Dever de casa 04 -- Ondas e Marés}
\author{Prof. Filipe Fernandes}
\date{20-Setembro-2013}

\begin{document}
\maketitle
\doublespacing

\vspace{1cm}
\hbox to \textwidth{Nome e número de matrícula:\enspace\hrulefill}
\vspace{1cm}

\begin{minipage}{.8\textwidth}
Essa lista incluí \numquestions\ questões. O número total de pontos é \numpoints.
\end{minipage}

% 10
\begin{questions}
\question
Definições.

 \begin{parts}
  \part[1]
  Defina e escreva as equações/relações para $\lambda$, $T$, $\omega$, $K$, $C$
  e $C_g$.

  \begin{solution}
  \begin{itemize}
    \item $\lambda \rightarrow$ Comprimento de onda.
    \item $T \rightarrow$ Período da onda.
    \item $\omega \rightarrow$ Frequência da onda.
    \item $K \rightarrow$ Número de onda.
    \item $C \rightarrow$ Velocidade de fase.
    \item $C_g \rightarrow$ Velocidade de grupo.
  \end{itemize}
  \(
  \omega = \dfrac{2\pi}{T},
  \)
  \(
  K = \dfrac{2\pi}{\lambda},
  \)
  \(
  C = \dfrac{\lambda}{T} = \dfrac{\omega}{K},
  \)
  \(
  C_g = \dfrac{C}{2} \text{ (curtas) ,}
  \)
  \(
  C_g = C \text{ (longas).}
  \)
  \end{solution}

  \part[1]
  Crie uma tabela de comparação entre ondas de gravidade internas, externas,
  ondas capilares, ondas de águas rasas e profundas.  Em sua tabela mencione as
  semanas e diferenças comentando as forçantes, trajetória das partículas e
  velocidade de propagação
  quando aplicável.

  \begin{solution}
  \raggedright
  a.) Ondas de gravidade de superfície e ondas capilares.\\
  Forçante vento | tensão superficial.\\
  Ambas ocorrem na interface ar-mar.\\

  b.) Ondas de água rasa e ondas de água profunda.\\
  Velocidade de fase metade da de grupo | igual a de grupo.\\
  Ambas são ondas de gravidade de superfície.\\

  c.) Ondas interna e onda externa.\\
  Interface de densidade | interface ar-mar\\
  Ambas força restauradora é a gravidade.
  \end{solution}
 \end{parts}

  \question
  Sobre ondas longas (água rasa) e curtas (água profunda) responda.

\begin{parts}
  \part[1] Escreva a relação de dispersão de ondas de gravidade de superfície.
  Fale sobre as aproximações feitas para se chegar nela.  Comente também
  como a forma da $\tanh$, presente na equação, pode nos ajudar a simplificar
  ainda mais a mesma para dois casos específicos.

  \begin{solution}
  \raggedright
  \(
    \omega^2 = gK\tanh(KH)
  \)
    \begin{itemize}
      \item Longe o sítio da forçante do vento.
      \item Sem atrito.
      \item Período da onda muito menor que o período inercial.
      \item Assume-se um estado médio e perturbações sobre esse (as ondas).
      \item A amplitude da onda é pequena quando comparada com a coluna d'água.
    \end{itemize}

  Água rasa (ondas longas) $KH << 1 \therefore \tanh(KH) \sim KH$\\
  Água profundas (ondas curtas) $KH >> 1 \therefore \tanh(KH) \sim 1$\\

  \end{solution}

  \part[1]
  Faça a aproximação para água rasa e água profunda e crie uma tabela com as
  equações para $\omega$, $T$, $K$, $\lambda$, $C$ e $Cg$ para cada um dos
  regimes de ondas.

  \begin{solution}
  \raggedright
      Água rasa (ondas longas):\\
      Dispersão: $\omega^2 = gK^2H$\\
      Velocidade de Fase: $C = \dfrac{\lambda}{T} = \dfrac{\omega}{K} = \sqrt{gH}$\\
      Velocidade de Grupo: $Cg = \pd{}{K}\left(\omega = K\sqrt{gH}\right) = C_g = \sqrt{gH}$\\

      Água fundas (ondas curtas):\\
      Dispersão: $\omega^2 = gK$\\
      Velocidade de Fase: $C = \dfrac{\lambda}{T} = \dfrac{\omega}{K} = \sqrt{g/K}$\\
      Velocidade de Grupo: $C_g = \pd{}{K}\left[\omega = g^{1/2}K^{1/2}\right] = C_g = \dfrac{1}{2}g^{1/2}K^{-1/2} = \dfrac{C}{2}$\\
  \end{solution}

  \part[2]
  Um terremoto na costa do Peru criou um série de ondas do tipo Tsunami com
  $\lambda >> 4$ km.  Estime quando tempo (em horas) levará para a primeira
  onda atingir o Japão.

  \begin{solution}
  $C = \sqrt{gH} = \sqrt{9.8 \times 4000} = 198$ m/s\\
  $10^7 m$ até o Japão, $50.5\times10^3$ s $\sim 14$ hrs
  \end{solution}

  \part[2]
  Um observador nota a chegada de um trem de ondas dispersivas de uma
  tempestade localizada a uma certa distância.  No tempo $t_1$ as ondas têm
  período de 10 s.  Num tempo $t_2 = t_1 + 3.6 \times 10^4$ s, as ondas têm um
  período de 8 s.

  \begin{itemize}
    \item[a)] Quando a tempestade ocorreu?
    \item[b)] Quão longe estava a tempestade?
    \item[c)] Quais eram os comprimentos de onda ($\lambda$) e número de onda
    ($k$) quando essas foram observadas em $t_1$?
  \end{itemize}

  \begin{solution}
  Temos que notar que (1) essas são ondas dispersivas (curtas) e (2) o ``trem''
  de ondas se propaga com a velocidade de grupo.
  \begin{align*}
    C &= \sqrt{\dfrac{g\lambda}{2\pi}}\\
    \dfrac{\lambda^2}{T^2} &= \dfrac{g\lambda}{2\pi}\\
    \lambda &= \dfrac{gT^2}{2\pi}\\
    C_g &= \dfrac{1}{2}C\\
    C_g &= \dfrac{1}{2}\dfrac{\lambda}{T}\\
    C_g &= \dfrac{gT}{4\pi}
  \end{align*}

  Resolvendo o sistema:\\
  Em $t_1$( \(T_1 = 10\) s) e \(D = C_gt_1\)\\
  Em $t_2$( \(T_2 = 8\) s) e \(D = C_g(t_1 + 3.6 \times 10^4)\)\\
  Resolvendo para $t_1$:\\
  $10t_1 = 8(t_1 + 3.6 \times 10^4) \therefore t_1 = 1.44 \times 10^5 \sim 40$  hr\\
  Resolvendo para $D$:\\
  $D = \dfrac{9.8 \times 10}{4\pi}{1.44 \times 10^5} \sim 1122$  km

%   $T_2 = 8$ s\\

  $\lambda_1 = \dfrac{9.8 \times 10^2}{2\pi} = 155.97$ m\\
  $k_1 = \dfrac{2\pi}{\lambda} = 0.04$ m$^{-1}$
  \end{solution}

\end{parts}

\question
 \begin{parts}
  \part[1]
  Descreva sobre como estudar ondas com observações. Sua resposta deve conter
  informações sobre:
  \begin{itemize}
    \item Altura significativa de ondas.
    \item Interação de ondas.
    \item Pista do vento (``fetch'') e estado do mar.
  \end{itemize}

  \begin{solution}
    Ondas de gravidade podem ser observadas usando réguas de me
  \end{solution}

  \part[1]
  Todo esse tempo estão assumindo nossas ondas como sendo.
  $$\eta = \eta_o \cos(kx - \omega t)$$.

  Explique o argumento da função coseno ($kx - \omega t$)

  A seguinte equação descreve a superfície livre de ondas de água rasa de
  gravidade.

  $$\dfrac{\partial^2 \eta}{\partial x^2} - \dfrac{1}{gH}\dfrac{\partial^2 \eta}
  {\partial t^2}$$

  Mostre que a forma de onda que escolhemos satisfaz essa equação.

  \begin{solution}
    \begin{align*}
      \dfrac{\partial^2}{\partial x^2}\left[ \eta_o\cos(kx-\omega t) \right] &=  \dfrac{1}{gH}\dfrac{\partial^2}{\partial t^2}\left[ \eta_o\cos(kx-\omega t) \right]\\
      -k\dfrac{\partial}{\partial x}\left[ \eta_o\sin(kx-\omega t) \right] &=  -\dfrac{\omega}{gH}\dfrac{\partial}{\partial t}\left[ \eta_o\sin(kx-\omega t) \right]\\
      -k^2\eta_o\cos(kx-\omega t) &= -\dfrac{\omega^2}{gH} \eta_o\cos(kx-\omega t)\\
      \dfrac{\omega^2}{gH} = k^2 &\therefore \omega^2 = gHk^2
    \end{align*}
  \end{solution}
 \end{parts}

\end{questions}
\end{document}
