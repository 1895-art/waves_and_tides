% Style.
\documentclass[letterpaper,portuguese,12pt,pdftex]{exam}

\usepackage{setspace}
\usepackage{lineno}
\usepackage[left=2.5cm,top=1.5cm,right=2.5cm]{geometry}

% Portuguese.
\usepackage[brazil]{babel}
\usepackage[T1]{fontenc}
\usepackage[utf8x]{inputenc}
\usepackage{textcomp}

% Font.
\usepackage{lmodern}

% Figures.
\usepackage{epsf,epsfig}

% Bibtex and extras.
\usepackage{natbib}
\usepackage{url}
\usepackage[bookmarks=false,colorlinks=true,urlcolor={green},linkcolor={green},pdfstartview={XYZ null null 1.22}]{hyperref}

% Math.
\usepackage{amssymb,amsmath}
\usepackage{mathtools}
\everymath{\displaystyle}

% Exam.
\addpoints
% \printanswers
\noprintanswers
\usepackage{color}
\definecolor{SolutionColor}{rgb}{0.8,0.9,1}
\shadedsolutions
\renewcommand{\solutiontitle}{\noindent\textbf{Solução:}\par\noindent}
\pagestyle{headandfoot}
\footer{}{Página \thepage\ de \numpages}{}
\boxedpoints
\pointsinrightmargin
\pointpoints{ponto}{pontos}
\hqword{Questão}
\hpword{Pontos}
\hsword{Nota}
% \qformat{\textbf{Question\thequestion}\quad(\thepoints)\hfill}

% User commands.
\newcommand{\pd}[2]{\dfrac{\partial #1}{\partial #2}}

% PDF metadata.
\pdfinfo{% hyperref overrides this
  /Title    (Exercício Extra -- Ondas e Marés)
  /Author   (Filipe Fernandes)
  /Creator  (Filipe Fernandes)
  /Producer (Filipe Fernandes)
  /Subject  (HW)
  /Keywords (HW, oceanografia)
}

% Front page.
\title{Exercício Extra -- Ondas e Marés}
\author{Prof. Filipe Fernandes}
\date{19-Novembro-2013}

\begin{document}
\maketitle
\doublespacing

\begin{minipage}{.8\textwidth}
Essa lista vale \numpoints{} pontos extra na nota final.
\end{minipage}

\begin{questions}
\question
Você está viajando em um barco na direção paralela à propagação do que parece
ser ondas lineares no oceano.  Enquanto você observa as cristas passarem pelo
navio você nota 7 cristas (ou seja, 6 ondas) passarem num intervalo de 3
minutos.

Ao checar com a casa de máquinas você descobre que o navio está se movendo
numa velocidade de 5 nós (2.57 m s$^{-1}$), e a eco-sonda marca 10 m de profundidade
quase constante.

\begin{parts}
  \part[1]
  Com a informação acima calcule o período de onda relativo ao barco ($T'$).
  Escreva uma relação para o comprimento de onda relativo ($L'$). E calcule um
  valor para esse comprimento de onda ``percebido''.
  (Dica: $L'$ é o quão longe o navio viaja de uma crista a outra)

  \begin{solution}
    \begin{align*}
      T' &= \dfrac{3 \times 60\text{ s}}{6} \\
      T' &= 30 \text{ s} \\
      L' &= UT'  \\
      L' &= 2.57 \text{ m s}^{-1} \times 30 \text{ s} \\
      L' &= 77 \text{ m}
    \end{align*}
  \end{solution}

  \part[1]
  Crie uma expressão para o período real da onda (sobre um ponto de referência
  da Terra e não do barco).  Essa expressão deve ser em função de $T', L'$ e $L$
  (que ainda não sabemos).

  (Dica: Considere a fração de um comprimento de onda completo que se move
  durante o período percebido.)

  \begin{solution}

    \begin{center}
      \includegraphics[scale=0.8]{./boat_wave.png}
    \end{center}

    \begin{align*}
      \Delta T &= T_2 - T_1 \\
      \Delta X &= L - L' \\
      \dfrac{\Delta X}{\Delta T} &= \frac{L}{T} = \frac{L - L'}{T'} \\
      T &= T'\left(\frac{L}{L - L'}\right)
    \end{align*}
  \end{solution}

  \part[1]
  Usando a sua resposta acima, juntamente com a relação de dispersão, resolva
  o período de onda $T$.  Simplifique assumindo ondas  longas.  Quantas
  soluções válidas existem?

  (Dica: Considere a fração de um comprimento de onda completo que se move
  durante o período percebido.)

  \begin{solution}
    \begin{align*}
      \omega^2 &= k^2hg \\
      \left( \dfrac{2\pi}{T} \right)^2 &= \left( \dfrac{2\pi}{L} \right)^2 gh \\
      L &= T\sqrt{gh}
    \end{align*}

    Combinando com o resultado acima:

    \begin{align*}
      T &= T'\pm \frac{L'}{\sqrt{gh}} \\
      T &= 30\text{ s} \pm \frac{77\text{ m}}{\sqrt{9.8 \text{ m s}^{-2} 10\text{ m}}} \\
      T &= 30 \text{ s} \pm 7.77\text{ s} \\
      T &= 37.8 \text{ s ou } 22.2\text{ s}
    \end{align*}

    Com as informações fornecidas não há como inferir a direção de propagação
    dessas ondas.

  \end{solution}


  \part[1]
  Ache o comprimento de ondas associada a cada $T$.

  \begin{solution}
    \begin{align*}
      L &= T\sqrt{gh} \\
      L_1 &= 37.8\text{ s } \times 9.9\text{ m s}^{-1} = 374\text{ m} \\
      L_2 &= 22.2\text{ s } \times 9.9\text{ m s}^{-1} = 220\text{ m}
    \end{align*}
  \end{solution}

  \part[1]
  Avalia a sua solução sobre a consideração de ondas longas que fizemos.  A
  solução é válida?

  \begin{solution}
    Ambas são validas $\dfrac{h}{L} < 0.05$.
    \begin{align*}
      \frac{h}{L_1} = 0.027 \\
      \frac{h}{L_2} = 0.045
    \end{align*}
  \end{solution}

  \end{parts}

\end{questions}
\end{document}
